\chapter{Установившийся режим короткого замыкания}
\label{chap:5}

\section{Общие замечания}
\label{sec:5-1}


\section{Основные характеристики и параметры}
\label{sec:5-2}


\section{Приведение цепи ротора к статору}
\label{sec:5-3}


\section{Влияние и учет нагрузки}
\label{sec:5-4}


\section{Расчет при отсутствии автоматического регулирования возбуждения}
\label{sec:5-5}


\section{Влияние автоматического регулирования возбуждения}
\label{sec:5-6}

Снижение напряжения, вызванное коротким замыканием. приводит в действие АРВ генераторов, и их возбуждение соответственно возрастает. Поэтому можно заранее предвидеть, что токи и напряжения при этих условиях всегда больше, чем при отсутствии АРВ. Степень такого увеличения зависит от удаленности короткого замыкания и параметров самих генераторов.

В самом деле, если при относительно удаленном коротком замыкании для восстановления напряжения генератора до нормального достаточно лишь немного увеличить возбуждение, то по мере уменьшения удаленности для этого, очевидно, требуется все большее возбуждение. Однако рост последнего у генератора ограничен известным пределом $ I_{f\text{пр}} $.

Следовательно, для каждого генератора можно установить наименьшую величину внешней реактивности, при коротком замыкании за которой генератор при предельном возбуждении обеспечивает нормальное напряжение на своих выводах. Такую реактивность назовем \so{критической реактивностью} $ x_{\text{кр}} $, a связанный с ней очевидным равенством ток

\begin{equation}
    \label{eq:5-16 I_kr}
    I_{\text{кр}}=\frac{U_{\text{н}}}{x_{\text{кр}}}
\end{equation}

--- \so{критическим током}.

Если внешняя реактивность меньше критической, то, несмотря на работу генератора с предельным возбуждением, его напряжение все равно остается ниже нормального. Когда же внешняя реактивность больше критической, то напряжение генератора достигает нормального значения при возбуждении, меньшем предельного.

Таким образом, при коротком замыкании генератор с АРВ в зависимости от внешней реактивности может работать только в одном из двух режимов -- \so{предельного возбуждения} или \so{нормального напряжения}. Лишь в частном случае, когда $ x_{\text{вн}} = x_{\text{кр}} $, оба режима существуют одновременно. Критерием для оценки возможности того или иного режима служит критическая реактивность, величина которой может быть определена по \colorbox{red}{(5-14)}, где следует положить $ E_q = E_{q\text{пр}} $, т.~е.

% остановился 2017-08-30

\section{Расчет при наличии автоматического регулирования возбуждения}
\label{sec:5-7}

В схеме с несколькими генераторами, ток от которых поступает по общим для них ветвям, понятие внешней реактивности по отношению к каждому из них уже теряет смысл. Поэтому здесь нельзя непосредственно использовать установленный в предыдущем параграфе критерий для однозначного определения возможного режима работы каждого генератора при рассматриваемом коротком замыкании. В данном случае расчет приходится вести путем последовательного приближения, задаваясь для генераторов с АРВ в зависимости от положения каждого из них относительно места короткого замыкания либо режимом предельного возбуждения (т.~е. вводя такой генератор в схему своими $ E_{q\text{пр}} $ и $ x_d $), либо режимом нормального напряжения (т.~е. принимая для такого генератора $ E=U_{\text{н}} $ и $ x=0 $) и делая затем проверку выбранных режимов. Последняя заключается в сопоставлении найденных для этих генераторов токов с их критическими токами. Для режима предельного возбуждения должно быть $ I \geqslant I_{\text{кр}} $  (или, иначе, $ U \leqslant U_{\text{н}} $), а для режима нормального напряжения $ I \leqslant I_{\text{кр}} $.

Если в результате проверки оказалось, что режимы некоторых генераторов выбраны неверно, то после их замены нужно сделать повторный расчет с последующей проверкой. При использовании расчетной модели такие пробы выполняются очень быстро. Однако и при аналитическом расчете в большинстве случаев удается с первого раза правильно выбрать режимы генераторов с АРВ. Для этого нужно внимательно проанализировать условия работы отдельных генераторов при рассматриваемом коротком замыкании. В первую очередь нужно установить возможный режим ближайшего к месту короткого замыкания генератора, и если оказывается, что для него должен быть принят режим предельного возбуждения, то следует перейти к оценке возможных режимов других генераторов (или станций), рассматривая их поочередно в порядке увеличения их удаленности. Как только выявлен генератор (или станция), находящийся в режиме нормального напряжения, все приключенные к нему элементы, которые не образуют пути для тока к месту короткого, могут быть отброшены. Это может существенно упростить схему.

Нагрузки увеличивают проводимость приключенной к генератору цепи и, как показано в примере \colorbox{green}{5-4}, могут влиять на режим его работы в условиях короткого замыкания. Это обстоятельство нужно учитывать при оценке возможного режима генераторов с АРВ.

Генераторы без АРВ вводят в схему, как обычно, своими реактивностями $ x_d $ и э.~д.~с. $ E_{q\text{пр}} $ которые у них были в предшествующем режиме. Наличие таких генераторов, вообще говоря, также может повлиять на режим работы генераторов с АРВ.

Все высказанные соображения наглядно иллюстрированы в приводимом ниже конкретном примере.