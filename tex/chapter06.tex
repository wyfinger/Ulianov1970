\chapter{Начальный момент внезапного нарушения режима}
\label{chap:6}

\section{Общие замечания}
\label{sec:6-1}

Прежде чем перейти к знакомству с общими уравнениями электромагнитного переходного процесса синхронной машины, рассмотрим сначала начальный момент такого процесса. Разумеется, все величины в начальный момент внезапного нарушения режима можно получить из упомянутых уравнений как их частное решение для $ t = 0$. Более того, поскольку индуктивности цепей исключают внезапное изменение тока, то значение последнего в начальный момент переходного процесса, вообще говоря, является известным: оно сохраняется таким, что и в конце заданного предшествующего режима. Однако при изменившихся условиях этот ток состоит уже из новых слагающих, которые возникают в данном переходном процессе.

Поскольку поставленная задача ограничена рассмотрением лишь начального момента, вращение ротора и обусловленное этим изменение индуктивностей машины, очевидно, не играют никакой роли. Другими словами, в данном случае машину можно рассматривать как трансформатор.

Исследование начального момента переходного процесса проще и нагляднее вести на основе принципа сохранения первоначального потокосцепления. В самом деле, коль скоро магнитный поток, сцепленный с ротором, в момент внезапного нарушения режима сохраняется неизменным, то соответствующая ему э.~д.~с., наведенная в статоре, в тот же момент также остается неизменной. Следовательно, для синхронной машины условия в начальный момент переходного процесса аналогичны тем же условиям для трансформатора, питаемого источником синусоидального напряжения.

Таким образом, можно предвидеть, что при переходном процессе ток статора синхронной машины состоит из двух слагающих, а именно: \so{периодической}, которая вызывается э.~д.~с., наводимой потоком ротора, и \so{апериодической}, обусловленной изменением потока статора.

Часто рассматривают внезапное изменение тока, имея в виду изменение лишь одной из его слагающих. При этом другие слагающие обеспечивают в момент нарушения режима сохранение предшествующего мгновенного значения тока.

Во всех дальнейших выкладках (как в данной главе, так и в последующих главах) условимся считать:

% TO-DO: этот список должен быть буквенным
\begin{enumerate} 
	\item
	продольную составляющую тока статора положительной, когда создаваемая ею н.~с. совпадает по направлению с н.~с. тока возбуждения;
	\item
	поперечную составляющую тока статора положительной, когда создаваемая ею н.~с. отстает на 90\textdegree (электрических) от н.~с. тока возбуждения; при наличии на роторе поперечного контура это же направление принимается положительным для его магнитной оси;
	\item
	все величины ротора приведенными к статору, причем они, как и все величины статора, выражены в относительных единицах.
\end{enumerate}

Установим теперь, какими э.~д.~с. и реактивностями можно характеризовать синхронную машину в начальный момент переходного процесса.

\section{Переходные э.~д.~с. и реактивности синхронной машины}
\label{sec:6-2}

Обратимся к балансу магнитных потоков в продольной оси ротора синхронной машины при установившемся симметричном режиме ее работы с отстающим по фазе током (\colorbox{red}{рис.~6-1,~а}). При отсутствии насыщения каждый из потоков и их отдельные составляющие можно рассматривать независимо один от другого. Так, полный поток обмотки возбуждения $ \overset{\;.}{\Phi}_f $, который был бы при холостом ходе машины, состоит из полезного потока $ \overset{\;.}{\Phi}_{fad} $ и потока рассеяния $ \overset{\;.}{\Phi}_{\sigma f} $. В свою очередь полезный поток $ \overset{\;.}{\Phi}_{fad} $ является геометрической разностью продольного потока в воздушном зазоре $ \overset{\;.}{\Phi}_{\sigma d} $ и потока продольной реакции статора $ \overset{\;.}{\Phi}_{ad} $. Результирующий магнитный поток $ \overset{\;.}{\Phi}_{f\sum} $, сцепленный с обмоткой возбуждения, складывается из потока $ \overset{\;.}{\Phi}_{\sigma d} $ и потока рассеяния $ \overset{\;.}{\Phi}_{\sigma f} $.

Рассмотрим, как изменится этот баланс, если предположить внезапное изменение, например увеличение потока продольной реакции статора на $ \Delta \overset{\;.}{\Phi}_{ad/0/} $. При этом будем считать, что кроме обмотки возбуждения никаких других контуров в продольной оси ротора не имеется.

В соответствии с законом Ленца приращение потока $ \Delta \overset{\;.}{\Phi}_{ad/0/} $ вызовет ответную реакцию обмотки возбуждения $ \Delta \overset{\;.}{\Phi}_{f/0/} $, причем приращения потокосцеплений $ \Delta \overset{\;.}{\Psi }_{ad/0/} $ и $ \Delta \overset{\;.}{\Psi }_{ad/0/} $ должны компенсировать друг друга, т.~е.

\section{Сверхпереходные э.~д.~с. и реактивности синхронной машины}
\label{sec:6-3}


\section{Сравнение реактивностей синхронной машины}
\label{sec:6-4}


\section{Характеристики двигателей и нагрузки}
\label{sec:6-5}


\section{Практический расчет начального сверхпереходного и ударного токов}
\label{sec:6-6}