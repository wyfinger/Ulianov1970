\chapter{Форсировка возбуждения и развозбуждение синхронной машины}
\label{chap:8}

\section{Общие замечания}
\label{sec:8-1}

Одной из наиболее эффективных и в то же время простых мер обеспечения надежности работы синхронных машин в большинстве аварийных условий является быстрое повышение их возбуждения или, как говорят, \so{быстродействующая форсировка возбуждения}. В зависимости от принятой системы возбуждения эффективность форсировки различна, что обусловливается особенностями выполнения каждой системы возбуждения. Это различие проявляется в возможных предельных величинах (потолках) токов возбуждения, а также в величинах скоростей нарастания тока возбуждения (принужденного).

Исследование переходного процесса при форсировке возбуждения в общем виде с учетом всех влияющих факторов очень сложно и практически выполнимо лишь с применением современной вычислительной техники. Существенное влияние на форсировку возбуждения оказывает насыщение магнитных систем как самой синхронной машины, так и элементов системы возбуждения. Это обстоятельство делает данную задачу нелинейной со всеми вытекающими отсюда затруднениями.

Несмотря на высказанные замечания, все же представляется целесообразным, даже на базе ранее принятых допущений (см.~§\ref{sec:2-1 osnovnye dopushcheniia}), рассмотреть процесс форсировки возбуждения и понять главным образом физическую сущность происходящих при этом явлений. Свою задачу ограничим случаями, когда машина имеет обычную электромашинную или ионную систему возбуждения. Здесь уместно подчеркнуть, что выбор той или иной системы возбуждения требует всестороннего подхода с различных точек зрения при одновременном учете ряда требований общего и специального характера.

Анализ переходного процесса при развозбуждении или гашении магнитного поля синхронной машины относительно проще, хотя бы уже по той причине, что этот процесс происходит, как правило, после отключения машины от сети. При этом насыщение магнитной системы сказывается также заметно, но даже при пренебрежении им можно получить достаточно правильное представление о протекании такого процесса.

В дальнейшем, так же как и в \ref{chap:7}{гл.~7}, предполагается, что все величины цепей ротора приведены к статору и выражены в системе относительных единиц. Для упрощения записи специальные обозначения, указывающие такое приведение, опущены.

\section{Включение обмотки возбуждения на постоянное напряжение}
\label{sec:8-2}


\section{Форсировка возбуждения при электромашинном возбудителе}
\label{sec:8-3}


\section{Форсировка при управляемых ионных и тиристорных системах возбуждения}
\label{sec:8-4}

В последнее время широкое применение находят ионные и тиристорные системы возбуждения\footnote{Ведутся работы по созданию бесщеточных систем возбуждения; их динамические характеристики находятся в стадии исследования.}; при этом используют управляемые ионные или тиристорные выпрямители.

Ионные и тиристорные системы возбуждения позволяют легко обеспечить при форсировке очень быстрое нарастание напряжения возбуждения и большую предельную величину последнего. Это достигается обычно установкой двух выпрямителей, включенных параллельно. Один из них обеспечивает возбуждение машины в нормальном режиме, а другой служит для форсировки возбуждения. Регулирование возбуждения машины в нормальных условиях производят, используя систему управления выпрямителей.

Поскольку ионные и тиристорные системы возбуждения практически безынерционны ($ T_e \approx 0,02\text{~сек.} $), можно считать, что при форсировке возбуждения напряжение на кольцах обмотки возбуждения синхронной машины возрастает до предельного $ u_{f\text{пр}} $ скачком. Поэтому все выражения, полученные ранее для форсировки возбуждения при электромашинном возбудителе, применимы и при указанных системах возбуждения, для чего достаточно положить в них $ T_e = 0 $; это приводит к значительному их упрощению.

\section{Гашение магнитного поля}
\label{sec:8-5}