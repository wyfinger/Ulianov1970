\chapter{Внезапное короткое замыкание синхронной машины}
\label{chap:9}

\section{Общие замечания}
\label{sec:9-1}

Анализ электромагнитного переходного процесса при внезапном коротком замыкании, рассматриваемый в настоящей главе, ограничен условием, что синхронная машина работает отдельно от других источников питания. Внешняя цепь ее статора при возникшем коротком замыкании характеризуется некоторым постоянным сопротивлением, преимущественно индуктивным.

Чтобы иметь некоторое представление о взаимном влиянии машин на характер протекания электромагнитного переходного процесса (при неизменной скорости их вращения), в конце главы данный вопрос кратко освещен для простейших условий, когда в схеме имеются две машины, связанные между собой через произвольные реактивности.

Вначале рассматривается переходный процесс в синхронной машине без демпферных обмоток и при отключенном устройстве автоматического регулирования возбуждения. В дальнейшем введен учет такого регулирования, используя материал предыдущей главы. Влияние и

учет демпферных обмоток изложен без строгих математических выкладок; при этом основное внимание обращено на вскрытие физической сущности явления и возможности упрощенной оценки этого влияния.

Практический интерес представляет протекание процесса при каскадном (или ступенчатом) отключении короткого замыкания и его повторном включении. В общем виде данный вопрос очень сложен. Поэтому здесь он рассмотрен применительно к условиям, когда в схеме имеется лишь одна машина.

\section{Внезапное короткое замыкание синхронной машины без демпферных обмоток}
\label{sec:9-2}


\section{Влияние и приближенный учет демпферных обмоток}
\label{sec:9-3}

Общий путь исследования электромагнитного переходного процесса внезапного короткого замыкания синхронной машины с демпферными обмотками принципиально тот же, что и в предыдущем параграфе. Такая машина характеризуется операторными реактивностями в обеих осях ротора. Каждая дополнительная обмотка на роторе повышает порядок определителя системы уравнений, аналогичной (9-7) и (9-8). Так, если по осям $ d $ и $ q $ расположено по одной демпферной обмотке, то $ p $ в определителе уже достигает пятой степени. При этом решение характеристического уравнения, получающегося путем приравнивания определителя нулю, в общем виде невозможно. Достаточно близкое к действительности решение можно получить, так же как и при отсутствии демпферных обмоток, пренебрегая поочередно активными сопротивлениями цепей ротора и статора.

При таком решении корни характеристического уравнения $ p_1 $ и $ p_2 $ могут быть определены по (9-12), где вместо $ x'_d $ и $ x_q $ нужно ввести соответственно $ x''_d $ и $ x''_q $. Для нахождения значений $ T_a $ и $ x_2 $ должна быть сделана аналогичная замена в (9-13) и (9-14).


\section{Влияние автоматического регулирования возбуждения при внезапном коротком замыкании}
\label{sec:9-4}


\section{Каскадное отключение и повторное включение короткого замыкания}
\label{sec:9-5}


\section{Взаимное электромагнитное влияние синхронных машин при переходном процессе}
\label{sec:9-6}