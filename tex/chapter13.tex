\chapter{Схемы отдельных последовательностей}
\label{chap:13 skhemy_otdelnykh_posledovatelnostei}

\section{Общие замечания}
\label{sec:13-1 obshchie_zamechaniia}

При применении метода симметричных составляющих к расчету любого несимметричного режима или процесса одной из первоочередных задач является составление схем замещения в общем случае всех трех последовательностей (прямой, обратной и нулевой).

При аналитическом решении поставленной задачи по этим схемам находят результирующие сопротивления отдельных последовательностей рассматриваемой системы относительно места, где возникла несимметрия. Из схемы замещения прямой последовательности, помимо того, находят результирующую э.~д.~с. относительно той же точки.

Когда используют расчетные модели (или столы), надобность в определении таких результирующих величин отпадает, так как в этом случае после объединения схем замещения отдельных последовательностей в комплексную схему для рассматриваемого вида несимметрии (см. \colorbox{red}{§ 14-7 и 15-6}) токи и напряжения отдельных последовательностей можно найти в некотором масштабе по показаниям измерительных приборов.

Схемы замещения отдельных последовательностей составляют, как обычно, в соответствии с указаниями \colorbox{red}{§ 2-4}. В частности, элементы схем замещения выражают в именованных или относительных единицах, приводя соответственно к выбранной основной ступени напряжения или к выбранным базисным условиям.

\section{Схемы прямой и обратной последовательности}
\label{sec:13-2 skhemy_priamoi_i_obratnoi_posledovatelnosti}

Схема прямой последовательности является обычной схемой, которую составляют для расчета любого симметричного трехфазного режима или процесса. В зависимости от применяемого метода расчета и момента времени в нее вводят генераторы и нагрузки соответствующими реактивностями и э.~д.~с. Все остальные элементы вводят в схему неизменными сопротивлениями.

Поскольку пути циркуляции токов обратной последовательности те же, что и токов прямой последовательности, схема обратной последовательности по структуре аналогична схеме прямой последовательности. Различие между ними состоит прежде всего в том, что в схеме обратной последовательности э.~д.~с. всех генерирующих ветвей условно принимают равными нулю (см. \colorbox{red}{§ 11-3}); кроме того, считают, что реактивности обратной последовательности синхронных машин и нагрузок практически постоянны и не зависят от вида и условий возникшей несимметрии, а также от продолжительности переходного процесса.

\so{Началом схемы прямой или обратной последовательности} считают точку, в которой объединены свободные концы всех генерирующих и нагрузочных ветвей; это точка нулевого потенциала схемы соответствующей последовательности.

\so{Концом схемы прямой или обратной последовательности} считают точку, где возникла рассматриваемая несимметрия. При продольной несимметрии каждая из схем имеет два конца; ими являются две точки, между которыми расположена данная продольная несимметрия. К концу или между концами схем отдельных последовательностей приложены напряжения соответствующих последовательностей, возникающие в месте несимметрии.

\section{Схема нулевой последовательности}
\label{sec:13-3 skhema_nulevoi_posledovatelnosti}

Токи нулевой последовательности по существу являются однофазным током, разветвленным между тремя фазами и возвращающимся через землю и параллельные ей цепи. В силу этого путь циркуляции токов нулевой последовательности резко отличен от пути, по которому проходят токи прямой или обратной последовательности.

Схема нулевой последовательности в значительной мере определяется соединением обмоток участвующих трансформаторов и автотрансформаторов.

\textit{Составление схемы нулевой последовательности следует начинать, как правило, от точки, где возникла несимметрия, считая, что в этой точке все фазы замкнуты между собой накоротко и к ней приложено напряжение нулевой последовательности.} В зависимости от вида несимметрии это напряжение прикладывается или относительно земли (поперечная несимметрия, \colorbox{red}{рис. 13-1,\textit{а}}) или последовательно, в рассечку фазных проводов (продольная несимметрия, \colorbox{red}{рис. 13-1,\textit{6}}).

Исходя из соответствующего данной несимметрии включения напряжения нулевой последовательности, далее следует выявить в пределах каждой электрически связанной цепи возможные пути протекания токов нулевой последовательности.

Когда напряжение нулевой последовательности приложено относительно земли, то при отсутствии емкостной проводимости для циркуляции токов нулевой последовательности необходима по меньшей мере одна заземленная нейтраль в той же электрически связанной цепи, где приложено это напряжение. При нескольких заземленных нейтралях в этой цепи образуется соответственно несколько параллельных контуров для токов нулевой последовательности.

При продольной несимметрии, т. е. когда напряжение нулевой последовательности введено последовательно, в фазные провода, циркуляция токов нулевой последовательности возможна даже при отсутствии заземленных нейтралей, если при этом имеется замкнутый контур через обходные пути той же электрически связанной цепи \footnote{При этом в земле циркулирует наведенный ток, следуя по трассе линии.}. При отсутствии таких путей протекание токов нулевой последовательности в рассматриваемых условиях возможно только в том случае, если в той же электрически связанной цепи имеются заземленные нейтрали с обеих сторон от места, где приложено напряжение нулевой последовательности.

\textit{Сопротивление, через которое заземлена нейтраль трансформатора, генератора, двигателя, нагрузки, должно быть введено в схему нулевой последовательности утроенной величиной.} Это обусловлено тем, что схему нулевой последовательности составляют для одной фазы, а через указанное сопротивление протекает сумма токов нулевой последовательности всех трех фаз.

Участие трансформаторов и автотрансформаторов в схеме нулевой последовательности достаточно подробно было рассмотрено в \colorbox{red}{§ 12-5} и \colorbox{red}{12-6}. В частности, сопротивление, введенное в нейтраль автотрансформатора, участвует в схеме замещения нулевой последовательности согласно \colorbox{red}{рис. 12-5,г}; реактивности этой схемы находят по \colorbox{red}{(12-11)}.

На \colorbox{red}{рис. 13-2} показан пример составления схемы нулевой последовательности для случая, когда напряжение нулевой последовательности возникает между проводами и землей (поперечная несимметрия). Стрелками указаны пути циркуляции токов нулевой последовательности при рассматриваемых условиях. Обмотки трансформаторов, автотрансформатора и прочие элементы схемы \colorbox{red}{рис. 13-2,о} обозначены порядковыми номерами, которые сохранены в обозначениях элементов схемы нулевой последовательности.

Поскольку в цепи среднего напряжения автотрансформатора имеется путь для токов нулевой последовательности, автотрансформатор входит своей полной схемой замещения. Циркуляция тока нулевой последовательности в обмотке \textit{12} трансформатора \textit{Т-2} обеспечена через заземленную нейтраль нагрузки. Этот трансформатор предполагается трехстержневым, поэтому учтена его реактивность намагничивания нулевой последовательности. Для другого трансформатора и автотрансформатора указания об их конструкции практически не нужны, так как они имеют обмотки, соединенные треугольником.

Если предположить, что в той же точке напряжение нулевой последовательности приложено в рассечку проводов, то легко убедиться, что в этом случае схема нулевой последовательности останется той же, но ее результирующее сопротивление будет совсем иным (см. \colorbox{red}{§ 13-4}).

В \colorbox{red}{§ 12-7} уже указывалось, что взаимоиндукция нулевой последовательности между параллельными цепями воздушных линий может сказываться весьма существенно. Поэтому ее нужно учитывать при составлении схемы нулевой последовательности, вводя такие цепи соответствующими схемами замещения. В приложении \colorbox{red}{П-8} приведен ряд схем замещения нулевой последовательности для нескольких типовых случаев, где требуется учет взаимоиндукции между цепями.

\so{Началом схемы нулевой последовательности} считают точку, в которой объединены ветви с нулевым потенциалом, а ее концом — точку, где возникла несимметрия. При продольной несимметрии схема нулевой последовательности имеет два конца (границы места несимметрии); при этом следует отметить, что когда нейтраль системы не заземлена, начало схемы уже теряет смысл, так как в общем случае точка нулевого потенциала может перемещаться в зависимости от характера продольной несимметрии, места ее возникновения и других факторов.

\section{Результирующие э.~д.~с. и сопротивления}
\label{sec:13-4 rezultiruiushchie_e.d.s._i_soprotivleniia}

Следующий этап аналитического расчета какого-либо несимметричного режима или процесса обычно заключается в определении результирующих сопротивлений схем отдельных последовательностей относительно точки, где возникла та или иная несимметрия. Помимо того, на этом этапе из схемы прямой последовательности находят также результирующую э.~д.~с. относительно той же точки. Необходимые для этого преобразования схем производят в соответствии с указаниями §~\ref{sec:2-6 primenenie_printcipa_nalozheniia}. При этом нужно особо иметь в виду принципиальное различие в преобразовании схем при поперечной и продольной несимметриях.

Обратимся к конкретной схеме \colorbox{red}{рис. 13-3,а} и проследим на ней, в чем именно состоит это различие. Все элементы этой схемы пронумерованы и их номера сохранены для обозначения соответствующих элементов в схемах замещения отдельных последовательностей.

При поперечной несимметрии в точке \textit{М} схема замещения прямой последовательности имеет вид, представленный на \colorbox{red}{рис. 13-3,6}. Последовательно соединенные в ней элементы \textit{1} и \textit{2}, а также \textit{5} и \textit{6} обозначены соответственно номерами \textit{8} и \textit{9}. Для определения результирующих э.~д.~с. и сопротивления относительно точки \textit{М} достаточно заменить ветвь \textit{9} с $ E = 0 $ и ветвь, получаемую сложением элемента \textit{8} с параллельно соединенными элементами \textit{3} и \textit{4} и имеющую э.~д.~с. \textit{Е}, одной эквивалентной (\colorbox{red}{рис. 13-3,в}). Схема обратной последовательности и ее преобразование аналогичны, за исключением того, что в ней отсутствуют э.~д.~с. источников. Схему нулевой последовательности (\colorbox{red}{рис. 13-3,г}) также легко преобразовать путем последовательного и параллельного сложения ветвей\footnote{Здесь сопротивления элементов \textit{3} и \textit{4} в общем случае подсчитывают с учетом взаимоиндукции между цепями линии}.

Пусть теперь в точке \textit{М} возникла продольная несимметрия. В этом случае напряжение прямой последовательности в точке \textit{М} должно быть введено в рассечку цепи элемента \textit{4} (\colorbox{red}{рис. 13-3,д}). Для определения результирующих э.~д.~с. и сопротивления схемы относительно точки \textit{М} в данном случае необходимо вначале сложить последовательно элементы \textit{8} и \textit{9}. Затем образовавшуюся ветвь \textit{10} с э.~д.~с. \textit{Е} и ветвь \textit{3} (\colorbox{red}{рис. 13-3,е}) следует заменить эквивалентной, что даст искомую результирующую э.~д.~с. относительно точки \textit{М}, а для нахождении результирующего сопротивления относительно той же точки достаточно к сопротивлению полученной эквивалентной ветви прибавить сопротивление элемента \textit{4}.

Схема обратной последовательности аналогична схеме \colorbox{red}{рис. 13-3,д}; в ней лишь отсутствует э.~д.~с. источника. Ее результирующее сопротивление находится так же, как и схемы прямой последовательности.

В схему нулевой последовательности (\colorbox{red}{рис. 13-3,ж}) двухцепная линия введена своей трехлучевой схемой замещения с элементами \textit{11}, \textit{12} и \textit{13}, с тем чтобы учесть взаимоиндукцию между цепями, находящимися теперь в различных условиях. Для нахождения результирующего сопротивления схемы здесь нужно сопротивление элемента \textit{11} сложить параллельно с суммой сопротивлений элементов \textit{2}, \textit{13}, \textit{5} и \textit{7} (последний входит утроенной величиной) и затем прибавить сопротивление элемента \textit{12}.

Соотношения между величинами результирующих сопротивлений одноименной последовательности при поперечной и продольной несимметриях в одной и той же точке могут быть самыми различными в зависимости от характера схемы, места несимметрии и других факторов.

\section{Распределение и трансформация токов и напряжений}
\label{sec:13-5 raspredelenie_i_transformatciia_tokov_i_napriazhenii}

Фазные токи и напряжения при несимметричных режимах или процессах проще всего находить путем суммирования симметричных составляющих. Поскольку рассматриваемые трехфазные схемы (или устройства) предполагаются выполненными симметрично, распределение токов и напряжений каждой последовательности находят в схеме одноименной последовательности, руководствуясь известными правилами и законами распределения токов и напряжений в линейных электрических цепях.

Вследствие того, что схемы обратной и нулевой последовательностей являются пассивными и их элементы остаются неизменными в течение всего переходного процесса, часто представляется целесообразным использовать коэффициенты распределения (см.~§\ref{sec:2-6 primenenie_printcipa_nalozheniia}), принимая за единицу ток каждой последовательности в месте несимметричного повреждения. При поперечной и продольной несимметриях в одной и той же точке эти коэффициенты различны. Однако при разных видах несиммстрии одного характера (т.~е. или поперечной, или продольной), возникающей в одной и той же точке системы, они одинаковы.

При определении фазных величин за трансформаторами нужно иметь в виду, что токи и напряжения при переходе через трансформатор изменяются не только по величине, но и по фазе в зависимости от соединения его обмоток.

Обратимся к \colorbox{red}{рис. 13-4}, где приведена принципиальная схема трансформатора с соединением обмоток $ Y_{0}/\Delta-11 $. Если число витков фазных обмоток соответственно равны $ \omega_{Y} $ и $ \omega_{\Delta} $, то линейный коэффициент трансформации (см.~§\ref{sec:2-4 sostavlenie skhemy zameshcheniia})

\begin{equation*}
	k = \sqrt{3}~\omega_{Y} / \omega_{\Delta}.
\end{equation*}

При заданных фазных токах $ \overset{\;.}{I}_A $, $ \overset{\;.}{I}_B $, $ \overset{\;.}{I}_C $ в соответствии
с принятыми на \colorbox{red}{рис. 13-4} положительными направлениями для токов в линейных проводах за треугольником имеем:

\begin{equation} % 13-1
	\label{eq:13-1 sistem}
	\left.
	\begin{matrix}
        \overset{\;.}{I}_a = \overset{\;.}{I}_{a\Delta} - \overset{\;.}{I}_{b\Delta} = (\overset{\;.}{I}_{A} - \overset{\;.}{I}_{B}) \frac{\omega_{Y}}{\omega_{\Delta}} = \frac{\overset{\;.}{I}_{A} - \overset{\;.}{I}_{B}}{\sqrt{3})} \\ 
        \overset{\;.}{I}_b = \overset{\;.}{I}_{b\Delta} - \overset{\;.}{I}_{c\Delta} = (\overset{\;.}{I}_{B} - \overset{\;.}{I}_{C}) \frac{\omega_{Y}}{\omega_{\Delta}} = \frac{\overset{\;.}{I}_{B} - \overset{\;.}{I}_{C}}{\sqrt{3})} \\ 
        \overset{\;.}{I}_c = \overset{\;.}{I}_{c\Delta} - \overset{\;.}{I}_{a\Delta} = (\overset{\;.}{I}_{C} - \overset{\;.}{I}_{A}) \frac{\omega_{Y}}{\omega_{\Delta}} = \frac{\overset{\;.}{I}_{C} - \overset{\;.}{I}_{A}}{\sqrt{3})}
    \end{matrix}\right\}
\end{equation}

Эту запись можно видоизменить, выразив токи через их симметричные составляющие. Так, например, для тока $ \overset{\;.}{I}_a $ получим:

\begin{equation*} % 13-2
    \label{eq:13-2 I_a}
	\overset{\;.}{I}_a = \frac{(\overset{\;.}{I}_{A1} + \overset{\;.}{I}_{A2} + \overset{\;.}{I}_0 - a^2\overset{\;.}{I}_{A1} - a\overset{\;.}{I}_{A2} - \overset{\;.}{I}_0)}{\sqrt{3}}\;k =
\end{equation*}
\begin{equation*}
    = \frac{(1 - a^2) \overset{\;.}{I}_{A1} + (1 - a) \overset{\;.}{I}_{A2}}{\sqrt{3}}\;k =
\end{equation*}    
\begin{equation}	
	= (\overset{\;.}{I}_{A1} e^{j30^\circ} + \overset{\;.}{I}_{A2} e^{-j30^\circ})\;k,
\end{equation}

откуда, в частности, видно, что, как и следовало ожидать, линейные токи за треугольником не содержат составляющих нулевой последовательности.

Аналогично могут быть найдены напряжения за рассматриваемым трансформатором. Если $ \overset{\;.}{U_A} $, $ \overset{\;.}{U_B} $ и $ \overset{\;.}{U_C} $  —фазные напряжения со стороны звезды, включающие в себя падения напряжения в самом трансформаторе\footnote{Эти напряжения по существу измерены за трансформатором, но приведены к стороне звезды, т.~е. $ U_A = \overset{\bullet}{U_a} $, $ U_B = \overset{\bullet}{U_a} $, $ U_C = \overset{\bullet}{U_c} $.}, то искомые фазные напряжения со стороны треугольника будут:

\begin{equation} % 13-3
    \label{eq:13-3 sistem}
    \left.
	\begin{matrix}
        \overset{\;.}{U}_a = \frac{\overset{\;.}{U}_A - \overset{\;.}{U}_B}{\sqrt{3}} \frac{1}{k} \\ 
        \overset{\;.}{U}_b = \frac{\overset{\;.}{U}_B - \overset{\;.}{U}_C}{\sqrt{3}} \frac{1}{k} \\ 
        \overset{\;.}{U}_c = \frac{\overset{\;.}{U}_C - \overset{\;.}{U}_A}{\sqrt{3}} \frac{1}{k}
    \end{matrix}\right\}
\end{equation}

или при выражении напряжений через симметричные составляющие, например, для напряжения $ \overset{\;.}{U}_a $ будем иметь:

\begin{equation} % 13-4
    \label{eq:13-4 U_a}
    \overset{\;.}{U_a} = (\overset{\;.}{U}_{A1} e^{j30^\circ} + \overset{\;.}{U}_{A2} e^{-j30^\circ}) \frac{1}{k}
\end{equation}

Из (\ref{eq:13-3 sistem}) и (\ref{eq:13-4 U_a}) следует, что напряжения на стороне треугольника не зависят от напряжения нулевой последовательности на стороне звезды. В то же время, если нейтраль системы на стороне треугольника смещена на $ \overset{\;.}{U}_{0\Delta} $, для определения фазных напряжений относительно земли к значениям по (\ref{eq:13-3 sistem}) и (\ref{eq:13-4 U_a}) нужно прибавить $ \overset{\;.}{U}_{0\Delta} $.

Структура (\ref{eq:13-2 I_a}) и (\ref{eq:13-4 U_a}) показывает, что при переходе со стороны звезды на сторону треугольника трансформатора, обмотки которого соединены по группе $ Y_{0}/\Delta-11 $, векторы прямой последовательности повертываются на $ 30^\circ $ в направлении вращения векторов, а векторы обратной последовательности — на $ 30^\circ $ в противоположном направлении (\colorbox{red}{рис. 13-5}).

При переходе через трансформатор в обратном направлении угловые смещения симметричных составляющих меняют свой знак на противоположный.

Наиболее простые соотношения получаются для трансформатора с соединением обмоток по группе 12, так как в этом случае угловые смещения токов и напряжений вообще отсутствуют. При этом, когда имеется соединение $ Y_{0}/Y_{0} $, должны быть учтены трансформируемые составляющие нулевой последовательности.

При нечетной группе соединения обмоток в тех случаях, когда не требуется знать истинной взаимной ориентировки векторных диаграмм на обеих сторонах трансформатора, можно для простоты считать его соединение по группе 3 (или 9), поскольку при этом векторы прямой и обратной последовательностей повертываются на 90\textdegree в противоположные направления (\colorbox{red}{рис. 13-6}). Очевидно, векторы прямой последовательности можно оставить без смещения, но векторы обратной последовательности сдвинуть на 190\textdegree. Отсюда вытекает простое и удобное для практики правило:

\textit{при переходе через трансформатор с соединением $ Y/\Delta $ (или $ \Delta/Y $) достаточно только у векторов обратной последовательности изменить знак на противоположный.}

Нужно иметь в виду, что отказ от учета действительной группы соединения обмоток трансформатора приводит к несовпадению обозначений линейных проводов за трансформатором с маркировкой, отвечающей действительной группе соединений.

Следует особо подчеркнуть, что если токи и напряжения выражены в относительных единицах, то при их трансформации должны учитываться только угловые сдвиги, обусловленные соответствующей группой соединения обмоток трансформатора.