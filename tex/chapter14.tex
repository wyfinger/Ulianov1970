\chapter{Однократная поперечная несимметрия}
\label{chap:14}

\section{Общие замечания}
\label{sec:14-1}

Поперечная несимметрия в произвольной точке трехфазной системы в общем виде может быть представлена присоединением в этой точке неодинаковых сопротивлений, как это, например, показано на \colorbox{red}{рис. 14-1.} Такой подход, вообще говоря, позволяет получить решение в общем виде, из которого затем вытекают решения для всех частных случаев. Однако решение в общем виде приводит к весьма громоздким выражениям\footnote{Даже при отсутствии взаимоиндукции между $ Z_A $, $ Z_B $ и $ Z_C $}. Поэтому значительно проще и нагляднее проводить решение для каждого вида поперечной несимметрии, используя характеризующие его граничные условия.

В настоящей главе рассмотрены три основных вида несимметричных коротких замыканий (двухфазное, однофазное и двухфазное на землю); при этом вначале предполагается, что эти замыкания металлические, а затем приведены указания, как учесть переходные сопротивления (дуга и пр.), которые могут быть в месте аварийного замыкания.

Приводимые выкладки, естественно, предполагают, что в соответствии с \colorbox{red}{§ 11-3} рассматриваются только основные гармоники тока и напряжения, причем схемы отдельных последовательностей состоят лишь из реактивностей и приведены к элементарному виду относительно короткого замыкания, т.~е. найдены результирующая э.~д.~с. и результирующие реактивности $ X_{1\sum} $, $ X_{0\sum} $ и $ X_{2\sum} $.

При записи граничных условий примем, что фаза А находится в условиях, отличных от условий для двух других фаз, т.~е. она является, как говорят, \so{особой фазой}. За положительное направление токов (фазных и их симметричных составляющих) будем считать направление к месту короткого замыкания. Наконец, чтобы упростить запись, будем опускать индекс вида короткого замыкания, сохраняя его только в записи граничных условий и в окончательных результатах выводов.


\section{Двухфазное короткое замыкание}
\label{sec:14-2}


Запишем граничные условия для двухфазного короткою замыкания \colorbox{red}{(рис. 14-2,а)}:


\section{Однофазное короткое замыкание}
\label{sec:14-3}


\section{Двухфазное короткое замыкание на землю}
\label{sec:14-4}


\section{Учет переходного сопротивления в месте замыкания}
\label{sec:14-5}


\section{Правило эквивалентности прямой последовательности}
\label{sec:14-6}


\section{Комплексные схемы замещения}
\label{sec:14-7}


\section{Сравнение видов короткого замыкания}
\label{sec:14-8}


\section{Векторные диаграммы токов и напряжений}
\label{sec:14-9}


\section{Взаимное электромагнитное влияние синхронных машин при переходном процессе несимметричного короткого замыкания}
\label{sec:14-10}


\section{Применение практических методов к расчету переходного процесса при однократной поперечной несимметрии}
\label{sec:14-11}