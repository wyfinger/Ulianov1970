\chapter{Общие указания к выполнению расчетов}

\section{Основные допущения}

Как отмечалось выше, расчет электромагнитного переходного процесса в современной электрической системе с учетом всех имеющих место условий и факторов чрезвычайно сложен и практически невыполним. Поэтому, чтобы упростить задачу и сделать ее решение практически возможным, вводят ряд допущений. Последние зависят прежде всего от характера и постановки самой задачи. Те допущения, которые вполне пригодны при решении одной задачи, могут быть совершенно неприемлемыми при решении другой.

Каждый из практических методов расчета, электромагнитных переходных процессов, в частности процесса при коротком замыкании, основан на некоторых допущениях, касающихся преимущественно возможности использование упрощённых представлений об изменении свободных токов в сложных схемах с несколькими источниками, о разных способах учета автоматического регулирования возбуждения синхронных машин и т.~п. C ними читатель познакомится в ходе дальнейшего изложения материала. Здесь же остановимся только на тех основных допущениях, которые обычно принимают при решении большинства практических задач, связанных с определением токов и напряжений при электромагнитных переходных процессах. К  числу таких допущений следует отнести:

\begin{enumerate} 
	\item
	отсутствие насыщения магнитных систем. При этом все схемы оказываются линейными, расчет которых значительно проще; в частности, здесь могут быть использованны любые формы принципа наложения.
	\item
	Пренебрежение токами намагничивания трансформаторов и автотрансформаторов. Единственным исключением их этого допущения является случай, когда трехстержневой трансформатор с соединением обмоток $ Y_{0}/Y_{0} $ включен на напряжение нулевой последовательности (см.~\colorbox{red}{§12-5}).
	\item
	Сохранение симметрии трехфазное системы. Оня нарушается обычно лишь для какого-либо одного элемента, что происходит в результату его повреждения, или преднамеренно по специальным соображениям (см. гл.~\colorbox{red}{15}).
	\item
	Пренебрежение емкостными проводимостями. Это допущение обычно является, уместным и заметно не искажает результаты решения, если в рассматриваемой схеме нет продольной компенсации индуктивности цепи, а также дальних линий передач напряжением выше 220~\textit{кв}. При рассмотрении простых замыканий на землю (см.~§\colorbox{red}{17-2}) это допущение, разумеется, совсем непригодно, так как в данном случае ток замыкается именно через емкостные проводимости.
	\item
	Приближенный учет нагрузок. В зависимости от стадии переходного процесса нагрузку приближенно характеризуют некоторым постоянным сопротивлением, обычно чисто индуктивным (см.~\colorbox{red}{§5-4 и §6-5}).
	\item
	Отсутствие активных сопротивлений. Это допущение в известной мере условно. Оно приемлемо при определении начальных и конечных значений отдельных величин, характеризующих переходный процесс в основных звеньях высокого напряжения электрической системы; при этом приближенный учет активных сопротивлений находит отражение при оценке постоянных времени затухания свободных составляющих рассматриваемых величин. В тех же случаях, когда подобный расчет проводится для протяженной кабельной или воздушной сети с относительно небольшими сечениями проводников (особенно линии со стальными проводами), а также для для установок и сетей напряжением до 1~\textit{кв}, данное допущение непригодно (см.~\colorbox{red}{гл. 17}).
	\item
	Отсутствие качаний синхронных машин. Если задача ограничена рассмотрением лишь начальной стадии переходного процесса (т.~е. в пределах 0,1--0,2~\textit{сек} с момента нарушения режима до отключения повреждения), это допущение обычно не вносит заметной погрешности (особенно в токе в месте повреждения). Однако при возникновении существенных качаний или выпадении машин из синхронизма достаточно надежный результат может быть получен лишь с учетом (хотя бы приближенным) такого процесса (см.~\colorbox{red}{гл. 19}).
	
	
\end{enumerate}