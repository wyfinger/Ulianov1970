\chapter{Общие указания к выполнению расчетов}

\section{Основные допущения}

Как отмечалось выше, расчет электромагнитного переходного процесса в современной электрической системе с учетом всех имеющих место условий и факторов чрезвычайно сложен и практически невыполним. Поэтому, чтобы упростить задачу и сделать ее решение практически возможным, вводят ряд допущений. Последние зависят прежде всего от характера и постановки самой задачи. Те допущения, которые вполне пригодны при решении одной задачи, могут быть совершенно неприемлемыми при решении другой.

Каждый из практических методов расчета, электромагнитных переходных процессов, в частности процесса при коротком замыкании, основан на некоторых допущениях, касающихся преимущественно возможности использование упрощённых представлений об изменении свободных токов в сложных схемах с несколькими источниками, о разных способах учета автоматического регулирования возбуждения синхронных машин и т.~п. C ними читатель познакомится в ходе дальнейшего изложения материала. Здесь же остановимся только на тех основных допущениях, которые обычно принимают при решении большинства практических задач, связанных с определением токов и напряжений при электромагнитных переходных процессах. К  числу таких допущений следует отнести:

\begin{enumerate} 
	\item
	отсутствие насыщения магнитных систем. При этом все схемы оказываются линейными, расчет которых значительно проще; в частности, здесь могут быть использованны любые формы принципа наложения.
	\item
	Пренебрежение токами намагничивания трансформаторов и автотрансформаторов. Единственным исключением их этого допущения является случай, когда трехстержневой трансформатор с соединением обмоток $ Y_{0}/Y_{0} $ включен на напряжение нулевой последовательности (см.~\colorbox{red}{§12-5}).
	\item
	Сохранение симметрии трехфазное системы. Она нарушается обычно лишь для какого-либо одного элемента, что происходит в результату его повреждения, или преднамеренно по специальным соображениям (см. гл.~\colorbox{red}{15}).
	\item
	Пренебрежение емкостными проводимостями. Это допущение обычно является, уместным и заметно не искажает результаты решения, если в рассматриваемой схеме нет продольной компенсации индуктивности цепи, а также дальних линий передач напряжением выше 220~\textit{кв}. При рассмотрении простых замыканий на землю (см.~§\colorbox{red}{17-2}) это допущение, разумеется, совсем непригодно, так как в данном случае ток замыкается именно через емкостные проводимости.
	\item
	Приближенный учет нагрузок. В зависимости от стадии переходного процесса нагрузку приближенно характеризуют некоторым постоянным сопротивлением, обычно чисто индуктивным (см.~\colorbox{red}{§5-4 и §6-5}).
	\item
	Отсутствие активных сопротивлений. Это допущение в известной мере условно. Оно приемлемо при определении начальных и конечных значений отдельных величин, характеризующих переходный процесс в основных звеньях высокого напряжения электрической системы; при этом приближенный учет активных сопротивлений находит отражение при оценке постоянных времени затухания свободных составляющих рассматриваемых величин. В тех же случаях, когда подобный расчет проводится для протяженной кабельной или воздушной сети с относительно небольшими сечениями проводников (особенно линии со стальными проводами), а также для для установок и сетей напряжением до 1~\textit{кв}, данное допущение непригодно (см.~\colorbox{red}{гл. 17}).
	\item
	Отсутствие качаний синхронных машин. Если задача ограничена рассмотрением лишь начальной стадии переходного процесса (т.~е. в пределах 0,1--0,2~\textit{сек} с момента нарушения режима до отключения повреждения), это допущение обычно не вносит заметной погрешности (особенно в токе в месте повреждения). Однако при возникновении существенных качаний или выпадении машин из синхронизма достаточно надежный результат может быть получен лишь с учетом (хотя бы приближенным) такого процесса (см.~\colorbox{red}{гл. 19}).
\end{enumerate}

\section{Понятие о расчетных условиях}
\label{sec:poniatie o raschetnykh usloviiakh}

В соответствии с целевым назначением проводимого на практике расчета электромагнитного переходного процесса устанавливают исходные расчетные условия. Они весьма разнообразны и при решении разных задач могут быть даже противоположными.

Так, например, для выбора выключателя по условиям его работы при коротком замыкании должны быть определены соответствующие возможные наибольшие величины тока короткого замыкания. С этой целью исходят из предположения, что короткое замыкание происходит в то время, когда включено наибольшее число генераторов, что вид короткого замыкания такой, при котором ток достигает наибольшей величины, что короткое замыкание металлическое и что оно произошло непосредственно у выводов самого выключателя. Помимо того, здесь устанавливают расчетное время размыкания контактов выключателя и цикл производимых им операций (включение и отключение).

Для выбора трубчатого разрядника требуется знать не только наибольшую, но и возможную наименьшую величину тока короткого замыкания, для определения которой, разумеется, должны быть приняты совсем иные расчетные условия.

Большое разнообразие расчетных условий встречается при выполнении расчетов для выбора и настройки устройств релейной защиты и автоматики. В них устанавливаются исходные предшествующие режимы заданной системы, число и расположение заземленных нейтралей, виды повреждений, и последовательность отключения поврежденного участка и т.~п.

При решении вопроса гашения поля синхронной машины в качестве расчетного режима может быть как режим короткого замыкания, так и холостого хода.

Приведенные примеры показывают, сколь велико разнообразие расчетных условий. Обоснование расчетных условий для конкретных технических задач (с учетом вероятности отдельных факторов) является одним из важных вопросов соответствующих специальных дисциплин.

\section{Система относительных единиц}

Представление любых физических величин не в обычных для них соответствующих именованных единицах, а в относительных, безразмерных единицах позволяет существенно упростить некоторые теоретические выкладки и придать им более общий характер. Равным образом и в практических расчетах такое представление величин придает результатам большую наглядность и позволяет быстрее ориентироваться в порядке определяемых значений. Благодаря этому система относительных единиц широко используется, хотя на первый взгляд она может казаться несколько искусственной и даже излишней.

С выражением величин в относительных единицах (в долях или процентах) читатель уже встречался при изучении электрических машин, где реактивности обычно выражают в долях единицы, напряжения короткого замыкания трансформаторов --- в процентах, пусковые токи и моменты асинхронных двигателей --- в кратностях от их номинальных значений и т.~д. Теперь нам нужно познакомиться с системой относительных единиц в более широком аспекте, имея в виду использование ее при решении различных вопросов и задач для схем с произвольным числом всевозможных элементов.

Напомним, что под относительным значением какой-либо величины следует понимать ее отношение к другой одноименной величине, выбранной за единицу измерения. Следовательно, чтобы выразить отдельные величины в относительных единицах, нужно прежде всего выбрать те величины, которые должны служить соответственными единицами измерения, или, как говорят, установить базисные единицы (или условия).

Пусть за базисный ток и базисное междуфазное напряжение приняты некоторые произвольные величины $ I_{\text{б}} $ и $ U_{\text{б}} $. Тогда базисная мощность трехфазной системы, очевидно, будет:

\begin{equation} % 2-1
	\label{eq:chap2 S_baz}
	S_{\text{б}} = \sqrt{3}U_{\text{б}}I_{\text{б}}
\end{equation}

и базисное сопротивление

\begin{equation} % 2-2
	\label{eq:chap2 z_baz}
	z_{\text{б}} = \frac{U_{\text{б}}}{\sqrt{3}I_{\text{б}}}
\end{equation}

т.~е. оно подчинено закону Ома, чтобы обеспечить тождественную запись этого закона как в именованных, так и в относительных единицах.

Как видно, из четырех базисных единиц $ I_{\text{б}} $, $ U_{\text{б}} $, $ S_{\text{б}} $ и $ z_{\text{б}} $ только две могут быть выбраны произвольно, а две другие уже получаются из указанных соотношений. Фазные и междуфазные базисные напряжения, а также фазные и линейные базисные токи связаны между собой известными соотношениями для симметричной трехфазной системы. Следует особо подчеркнуть, что выбранные базисные единицы служат для, измерения как полных величин, так и их составляющих (активных, реактивных и~пр.).

Таким образом, при выбранных базисных условиях относительные значения э.~д.~с., напряжения, тока, мощности и сопротивления будут:

\begin{equation} % 2-3
	\label{eq:chap2 E_baz_otn}
	\underset{*}{E_{\text{(б)}}} = \frac{E}{U_{\text{б}}}
\end{equation}

\begin{equation} % 2-4
	\label{eq:chap2 U_baz_otn}
	\underset{*}{U_{\text{(б)}}} = \frac{U}{U_{\text{б}}}
\end{equation}

\begin{equation} % 2-5
	\label{eq:chap2 I_baz_otn}
	\underset{*}{I_{\text{(б)}}} = \frac{I}{I_{\text{б}}}
\end{equation}

\begin{equation} % 2-6
	\label{eq:chap2 S_baz_otn}
	\underset{*}{S_{\text{(б)}}} = \frac{S}{S_{\text{б}}}
\end{equation}

\begin{equation} % 2-7
	\label{eq:chap2 z_baz_otn}
	\underset{*}{z_{\text{(б)}}} = \frac{z}{z_{\text{б}}}
\end{equation}

где звездочка указывает, что величина выражена в относительных единицах, а индекс (б) --- что она приведена к базисным условиям. Эти индексы, как и многие другие, часто опускают, если смысл выражения ясен из текста.

Относительные фазные и междуфазные напряжения численно одинаковы; равным образом численно одинаковы относительные фазная мощность и мощность трех фаз.

Используя (\ref{eq:chap2 z_baz}), можно формальное определение относительного сопротивления по (\ref{eq:chap2 z_baz_otn}) представить в ином виде:

\begin{equation} % 2-8
	\label{eq:chap2 z_baz_otn 2}
	\underset{*}{z_{\text{(б)}}} = \frac{z}{z_{\text{б}}} = \frac{\sqrt{3}I_{\text{б}}z}{U_{\text{б}}}
\end{equation}

или, иначе,

\begin{equation} % 2-9
	\label{eq:chap2 z_baz_otn 3}
	\underset{*}{z_{\text{(б)}}} = z\frac{S_{\text{б}}}{U_{\text{б}}^{2}}
\end{equation}

где $ z $ --- заданное сопротивление, \textit{ом} на фазу;
$ I_{\text{б}} $ --- базисный ток, \textit{ка} (\textit{а});
$ U_{\text{б}} $ --- базисное междуфазное напряжение, \textit{кв} (\textit{в});
$ S_{\text{б}} $ --- базисная мощность, \textit{Мва} (\textit{ва}).

Из последних выражений следует, что относительное сопротивление численно равно относительному падению напряжения в данном элементе при протекании через него принятого базисного тока (или мощности).

Поскольку выбор базисных условий произволен, то одна и та же действительная величина может иметь разные численные значения при выражении ее в относительных единицах. Обычно относительные сопротивления элементов задаются при номинальных условиях (т.~е. при $ I_{\text{н}} $ или $ S_{\text{н}} $ и $ U_{\text{н}} $). Их величины определяются по (\ref{eq:chap2 z_baz_otn 2}) и (\ref{eq:chap2 z_baz_otn 3}), где базисные единицы должны быть заменены соответственными номинальными, т.~е.

\begin{equation} % 2-8а
	\label{eq:chap2 z_nom_otn}
	\underset{*}{z_{\text{(н)}}} = \frac{\sqrt{3}I_{\text{н}}z}{U_{\text{н}}} \tag{\ref*{eq:chap2 z_baz_otn 2}а}
\end{equation}

и

\begin{equation} % 2-9а
	\label{eq:chap2 z_nom_otn 2}
	\underset{*}{z_{\text{(н)}}} = z\frac{S_{\text{н}}}{U_{\text{н}}^{2}} \tag{\ref*{eq:chap2 z_baz_otn 3}а}
\end{equation}

Иногда относительные величины выражают не в долевых единицах, а в процентах. Связь между такими выражениями очевидна; так, например,

\begin{equation} % 2-10
	\label{eq:chap2 z_procent}
	z_\% = 100z
\end{equation}

Активное сопротивление трансформатора весьма мало. Поэтому, пренебрегая им, можно считать, что задаваемое в процентах напряжение короткого замыкания трансформатора $ U_{\text{к}\%} = z_\% \approx x_\% $ . Если при этом принять, что индуктивное сопротивление рассеяния трансформатора приближенно изменяется пропорционально квадрату числа витков его обмоток (что довольно близко к действительности), то заданное значение $ U_{\text{к}\%} $, следует считать от напряжения холостого хода того ответвления регулируемой обмотки, которое установлено у трансформатора.

Для выполнения расчета в относительных единицах нужно всё э.~д.~с. и сопротивления элементов схемы выразить в относительных единицах при выбранных базисных условиях. Если они заданы в именованных единицах, то для перевода их относительные единицы служат выражения (\ref{eq:chap2 E_baz_otn}), (\ref{eq:chap2 z_baz_otn 2}) или (\ref{eq:chap2 z_baz_otn 3}). Когда же они заданы в относительных единицах при номинальных условиях, то их пересчет к базисным условиям нужно производить по следующим очевидным соотношениям:

\begin{equation} % 2-11
	\label{eq:chap2 E_baz_from_E_nom}
	\underset{*}{E_{\text{(б)}}} = \underset{*}{E_{\text{(н)}}}\frac{U_{\text{н}}}{U_{\text{б}}}
\end{equation}

\begin{equation} % 2-12
	\label{eq:chap2 z_baz_from_z_nom}
	\underset{*}{z_{\text{(б)}}} = \underset{*}{z_{\text{(н)}}}\frac{I_{\text{б}}}{I_{\text{н}}}\frac{U_{\text{н}}}{U_{\text{б}}}
\end{equation}

или

\begin{equation} % 2-13
	\label{eq:chap2 z_baz_from_z_nom 2}
	\underset{*}{z_{\text{(б)}}} = \underset{*}{z_{\text{(н)}}}\frac{S_{\text{б}}}{S_{\text{н}}}\frac{U_{\text{н}}^2}{U_{\text{б}}^2}
\end{equation}

При выборе базисных условий следуем руководствоваться соображениями, чтобы вычислительная работа была по возможности проще и порядок числовых значений относительных базисных величин был достаточно удобен для оперирования с ними. Для базисной мощности $ S_{\text{б}} $, целесообразно принимать простое круглое число (1000~\textit{Мва}, 100~\textit{Мва} и т.~п.), а иногда часто повторяющуюся в заданной схеме номинальную мощность (или кратную ей). За $ U_{\text{б}} $ рекомендуется принимать $ S_{\text{н}} $ или близкое к нему. При $ U_{\text{б}} = U_{\text{н}} $ пересчет относительных э.~д.~с. вообще отпадает ($ \underset{*}{E_{\text{(б)}}} = \underset{*}{E_{\text{(н)}}} $), а выражения для пересчета относительных сопротивлений принимают более простой вид:

\begin{equation} % 2-12а
	\label{eq:chap2 z_baz_from_z_nom a}
	\underset{*}{z_{\text{(б)}}} = \underset{*}{z_{\text{(н)}}}\frac{I_{\text{б}}}{I_{\text{н}}} \tag{\ref*{eq:chap2 z_baz_from_z_nom}а}
\end{equation}

\begin{equation} % 2-13а
	\label{eq:chap2 z_baz_from_z_nom 2 a}
	\underset{*}{z_{\text{(б)}}} = \underset{*}{z_{\text{(н)}}}\frac{S_{\text{б}}}{S_{\text{н}}} \tag{\ref*{eq:chap2 z_baz_from_z_nom 2}а}
\end{equation}

Равенство $ U_{\text{б}} = U_{\text{н}} $ вообще говоря, соблюдается только для части элементов, так как напряжения $ U_{\text{н}} $ элементов одной и той же электрической цепи в общем случае могут быть неодинаковы. Однако это различие сравнительно мало (в пределах $ \pm $10\%) и в приближенных расчетах им часто пренебрегают, полагая $ U_{\text{н}} $ всех элементов одной ступени напряжения одинаковыми и равными некоторому среднему номинальному напряжению $ U_{\text{ср}} $ для этой цепи. (см. \colorbox{red}{§ 2-4}). Исключение целесообразно делать для реакторов, поскольку они составляют обычно значительную часть общего сопротивления цепи, определение которого всегда желательно производить с большей точностью. В тех случаях, когда реакторы использованы на напряжениях ниже их номинальных напряжений (например, реактор 10~\textit{кв} в установке 6~\textit{кв} и т.~п.), пересчет их относительных сопротивлений по напряжениям, конечно, обязателен.

















