\chapter*{Предисловие}
\addcontentsline{toc}{chapter}{Предисловие}
\label{chap:0 preface}

Предлагаемая книга является учебником по первой части курса «Переходные процессы в электрических системах», в которой рассматриваются только электромагнитные переходные процессы.

Она написана в соответствии с программой по данному курсу (инд. У-Т-3/160), утвержденной Учебно-методическим Управлением MB и ССО СССР в 1968~г. для специальностей: «Электрические станции» (0301), «Электрические системы и сети» (0302) и «Кибернетика электрических систем» (0304). С некоторыми сокращениями она, очевидно, может быть использована и для других электротехнических специальностей и специализаций.

Весь материал книги разбит на четыре раздела; при этом в четвертый раздел отнесены \colorbox{red}{гл.~16--19} которые между собой не связаны.

При построение книги автор опирался преимущественно на свой многолетний опыт преподавания данного курса в Московском ордена Ленина энергетическом институте. Следует отметить, что не весь материал подлежит изложению на лекциях. Так, например, содержание гл.~\ref{chap:2 obshchie ukazaniia k vypolneniiu raschetov} почти полностью целесообразно прорабатывать на практических занятиях. К тому же, это в сущности вынужденное решение, так как лектор не успевает, прочитать все, что нужно к первому практическому занятию.

В зависимости от местных условий и обстоятельств (как-то: наличие лаборатории по курсу и ее пропускной способности и пр.) в рабочем календарном плане иногда приходится менять порядок прохождения отдельных тем, добиваясь наибольшей согласованности с тематикой практических занятий и содержанием каждого этапа заданий, которые самостоятельно выполняют студенты. Для этого основы строгой теории переходных процессов и ее применение (гл. \colorbox{red}{7—9}) лектор обычно вынужден излагать после практических методов расчета (гл. \colorbox{red}{10}). Равным образом более подробное знакомство с гл. \colorbox{red}{13} приходится давать после гл. \colorbox{red}{14} и \colorbox{red}{15}. Однако сделать такую перестановку в учебнике было бы неправильным, так как местные условия могут быть весьма различны, а кроме того, учебником пользуются учащиеся, которые не ограничены подобными рамками (например, студенты-заочники).

Несмотря на то что недавно вышел в свет сборник задач по данной части курса, автор не счел возможным ограничиться малым числом примеров. Все принципиальные вопросы и методы расчета в книге иллюстрированы необходимым количеством примеров, в которых приведены подробные решения.

Автор надеется; что эта книга найдет своих читателей также среди инженерно-технических работников и принесет им пользу в их практической деятельности.

При создании данной книги автор использовал не только свои работы, но также многочисленные работы по исследованию и расчету электромагнитных переходных процессов, выполненные в Советском Союзе: А.~А.~Горева, Н.~Н.~Щедрина, Д.~А.~Городского, Н.~Ф.~Марголина, Л.~Г.~Мамиконянца, И.~М.~Марковича, А.~Б.~Чернина и др. и за рубежом: Р.~Рюденберга, К.~Парка, Э.~Кларк, К.~Вагнера, Р.~Эванса, Э.~Кимбарка, К.~Ковача, И.~Раца и др. Поскольку книга предназначена для учебных целей, не представляется возможным всюду давать ссылки на первоисточники. Помещенный в конце книги перечень литературы ориентирован в основном на интересы и возможности студентов. Более полный, но далеко не исчерпывающий список литературы приведен в книге автора, изданной в 1964 г. \colorbox{red}{[Л. 4]}.

Автор выражает глубокую благодарность коллективу кафедры «Электрические станции, сети и системы» Рижского политехнического института и доктору техн. наук, шроф. Н.~И.~Соколову за рецензирование рукописи и сделанные ими замечания и предложения, которые учтены при окончательной подготовке рукописи к печати.

С благодарностью автор отмечает большую работу канд. техн. наук, доц. И.~П.~Крючкова по тщательному редактированию рукописи.

Все замечания и пожелания по данной книге автор примет с (признательностью и просит их направлять в адрес издательства «Энергия» (Москва, Ж-114, Шлюзовая наб., 10).

\vspace{1pc}
\begin{minipage}{0.49\linewidth}
	Москва, 1970
\end{minipage}
\hfill
\begin{minipage}{0.49\linewidth}
	\flushright
	\textit{С.~А.~Ульянов}
\end{minipage}

\chapter*{Введение}
\addcontentsline{toc}{chapter}{Введение}
\label{chap:0 intro}

Курс «Переходные процессы в электрических системах» является одним из профилирующих для электроэнергетических специальностей и специализаций.

Переходные процессы возникают в электрических системах как при нормальной эксплуатации (включение и отключение нагрузок, источников питания, отдельных цепей, производство испытаний и пр.), так и в аварийных условиях (обрыв нагруженной цепи или отдельной ее фазы, короткое замыкание, выпадение машины из синхронизма и т.~д. Их изучение, разумеется, не может быть самоцелью. Оно необходимо прежде всего для ясного представления причин возникновения и физической сущности этих процессов, а также дли разработки практических критериев и методов их количественной оценки, с тем чтобы можно было предвидеть и заранее предотвратить опасные последствия таких процессов. Короче говоря, важно понимать переходные процессы, но еще важнее уметь сознательно управлять ими.

При любом переходном процессе происходит в той или иной мере изменение электромагнитного состояния элементов системы и нарушение баланса между моментом на валу каждой вращающийся машины и электромагнитным моментом.

В результате этого нарушения соответственно изменяются скорости вращения машин, т.~е. некоторые машины испытывают торможение, в то время как другие --- ускорение. Такое положение существует до тех пор, пока регулирующие устройства не восстановят нормальное состояние, если это вообще осуществимо при изменившихся условиях.

Из сказанного следует, что переходный процесс характеризуется совокупностью электромагнитных и механических изменений в системе. Последние взаимно связаны и по существу представляют единое целое. Тем не менее благодаря довольно большой механической инерции вращающихся машин начальная стадия переходного процесса характеризуется преимущественно электромагнитными изменениями. В самом деле, вспомним хотя бы процесс пуска асинхронного двигателя. С момента включения его в сеть до момента начала разворота ротора двигателя имеет место только электромагнитный переходный процесс, который затем дополняется механическим переходным процессом. Процесс пуска двигателя значительно усложняется, если учесть возникающую реакцию источника питания и действие его автоматических регулирующих устройств.

При относительно малых возмущениях (например, при коротком замыкании за большим сопротивлением или, как говорят, при большой удаленности короткого замыкания) весь переходный процесс практически можно рассматривать только как электромагнитный. Для иллюстрации укажем, что в установке с напряжением 400~\textit{в} ток короткого замыкания в 5000~\textit{а} после его приведения к стороне генераторного напряжения составляет менее 1,5\% номинального тока современного турбогенератора 200~\textit{Мвт}(15,75~\textit{кв}). Естественно, такое малое увеличение тока не вызовет заметного нарушения равновесия рабочего состояния упомянутого турбогенератора.

Таким образом, при известных условиях представляется возможным и целесообразным рассматривать только одну   сторону переходного процесса, а именно явления электромагнитного характера. В  соответствии с этим настоящий курс разбит на две части. В первой из них рассматриваются электромагнитные переходные процессы\footnote{В конце первой части рассматривается упрощенный учет качаний генераторов, что является естественным переходом ко второй части курса.}, а во второй --- совместно электромагнитные и механические, т.~е. электромеханические переходные процессы. Такое деление помогает учащемуся постепенно осваивать разнообразный и достаточно сложный материал курса.

При провождении курса «Теоретические основы электротехники» читатель уже знакомился с переходными процессами в цепях с сосредоточенными и распределенными параметрами. Рассмотрение этих процессов проводилось в предположении, что цепь является однофазной и ее питание осуществляется от источника с заранее известным напряжением (как по величине, так и по закону его изменения). В данном курсе предстоит рассмотреть более сложные задачи, когда переходный процесс возникает в многофазной цепи, при этом он одновременно протекает в самих источниках питания, у которых дополнительно приходят в действие автоматические регулирующие устройства. В этом случае напряжения всех источников\footnote{За исключением тех, мощность которых практически может быть принята бесконечно большой.} являются неизвестными переменными величинами.

Преподавание в вузах этого курса как самостоятельной специальной дисциплины\footnote{Точнее, двух дисциплин,  так как вначале читались отдельно курс коротких замыканий и курс устойчивости электрических систем.} началось в конце 20-х годов. За истекшее время его содержание и число часов, отводимое на него в учебных планах, неоднократно менялось. В последние годы установлена более тесная последовательная связь между его обеими частями.

Первая часть данного курса использует материал, изученный в курсах высшей математики (операционное исчисление), теоретических основ электротехники (линейные цепи), электрических машин (преимущественно синхронные и асинхронные машины) и электрических сетей и систем.

В свою очередь материал первой части данного курса используется при прохождении его второй части, а также при дальнейшем изучении других специальных курсов, как-то: электрических систем, дальних передач, основного электрооборудования станций, техники релейной защиты, автоматизации электрических систем и др.

Практические задачи, при решении которых инженер-электрик сталкивается с необходимостью количественной оценки тех или иных величин во время электромагнитного переходного процесса, многочисленны и разнообразны (см.~§\ref{sec:1-3 naznacheniia_raschetov_i_trebovaniia_k_nim}). Однако все они в конечном итоге объединены единой целью обеспечить надежность работы отдельных элементов и электрической системы в целом.

Теперь сделаем небольшую экскурсию в прошлое и покажем вкратце как развивалась проблема переходных процессов преимущественно в части исследования электромагнитных переходных процессов.

В то время как теория установившихся режимов развивалась в правильном направлении и быстро приспособилась к нуждам практики, сущность переходных процессов долго оставалась невыясненной. На примере развития электромашиностроения нетрудно проследить, насколько важен учет явлений, в частности, при коротких замыканиях.

Первоначальные конструкции электрических машин выполнялись лишь в соответствии с требованиями нормальной работы. Пока мощности машин были малы, их конструкции обладали как бы естественным запасом устойчивости против механических и тепловых действий токов короткого замыкания. Однако такое положение существовало недолго. По мере роста мощности машин и особенно после осуществления их параллельной работы размер повреждений машин при коротких замыканиях резко возрос. Становилось очевидным, что нельзя обеспечить надежную конструкцию машины, не считаясь с аварийными условиями работы. Успех предлагаемых мер по усилению конструкций зависел от достоверности знаний самого процесса короткого замыкания. Так постепенно создавались все более совершенные конструкции электрических машин. В современном исполнении они являются одним из надежных элементов систем. Разумеется, эта надежность достигнута при учете и других опасных условий, в которых может оказаться машина.

Аналогичное положение наблюдалось при поисках способов гашения магнитного поля электрических машин. Недостаточность первоначальных сведений об этом процессе приводила к малоэффективным решениям. Подобные примеры можно обнаружить и в других областях электроэнергетики (аппаратостроении, технике релейной защиты и др.).

Более серьезная разработка теории переходных процессов в электрических машинах началась с первых лет текущего столетия. В конце 20-х годов Парк (Park) разработал строгую теорию переходных процессов в электрических машинах, приняв в основу ранее предложенную Блонделем (Blondel) теорию двух реакций. Эта теория обеспечила быстрое развитие дальнейших исследований в данной области. Они интенсивно проводились у нас в Союзе и за рубежом, главным образом в США. Особое место среди них занимают работы А.~А.~Горева.

Примерно в те же годы стала находить все более широкое применение теория симметричных составляющих, остававшаяся в течение нескольких лет без использования. Она позволила решить на строгой научной основе все вопросы, связанные с несимметрией в многофазной цепи.

Наряду с теоретическими исследованиями существенно важной являлась своевременная разработка практических методов расчета переходных процессов. В этом испытывалась острая нужда в связи с проводившейся широкой электрификацией нашей страны.

К выполнению таких работ привлекались научно-исследовательские и учебные институты (ВЭИ, МЭИ, ЛПИ, ХЭТИ и др.), крупные энергообъединения (Мосэнерго, Ленэнерго) и проектные организации (ТЭП). Для координации работ, обобщения результатов, подготовки решений и рекомендаций были созданы специальные комиссии. Так, в 30-х годах под председательством К.~А.~Круга работала комиссия по разработке указаний к выполнению расчетов коротких замыканий.

Теоретические исследования и практические методы расчета всегда требуют экспериментальной проверки. Ранее ее проводили в натуральных условиях. Однако испытания проводились крайне редко из-за значительного риска, что такой эксперимент повлечет серьезную аварию, поскольку системы не располагали достаточным резервом мощности, связи между станциями были слабы, отсутствовали многие автоматические устройства (как-то: регулирование возбуждения генераторов, повторное включение цепей и др.) и, наконец, само оборудование было еще недостаточно совершенным (например, время действия выключателей составляло десятые доли секунды). Позже и особенно в последнее время благодаря значительному усовершенствованию электрических систем подобные эксперименты проводят по мере надобности, причем, как правило, они не вызывают каких-либо заметных помех в нормальной работе системы. С той же целью используются записи автоматических осциллографов, которыми все больше оснащают наиболее ответственные и характерные цепи систем.

Неоценимую помощь в экспериментировании и проверке ряда новых теоретических разработок, схем и автоматических устройств оказало и продолжает оказывать физическое и математическое моделирование электрических систем. Применение электронных вычислительных машин непрерывного действия (машины-аналоги) и дискретного действия (цифровые машины) в значительной мере расширили возможности очень эффективного математического моделирования.

Расчетные модели, где все элементы системы (включая генераторы) представлены схемами замещения, уже свыше 35~лет широко используют для решения многих задач. В зависимости от их конструкции они позволяют получить решение в соответствии с принятым методом расчета, почти полностью освобождая от утомительной и трудоемкой вычислительной работы, что также очень ценно.

По вопросам переходных процессов в электрических системах, их моделированию и практическим методам их расчета написано много книг. Лишь некоторые из них указаны в данном учебнике.