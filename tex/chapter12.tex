\chapter{Параметры элементов для токов обратной и нулевой последовательностей}
\label{chap:12}

\section{Общие замечания}
\label{sec:12-1}

Все сопротивления, которыми характеризуются отдельные элементы в нормальном симметричном режиме, а также в симметричном переходном процессе, по существу являются сопротивлениями прямой последовательности\footnote{Исключение составляет реактивность, используемая при определении постоянной времени $ T_a $ (см~§ \ref{chap:9-2})}. Этот термин вводить ранее не было нужды, поскольку токи были лишь одной последовательности.

При отсутствии магнитной связи между фазами какого-либо элемента его сопротивление не зависит от порядка чередования фаз тока. Активная и реактивная слагающие сопротивления такого элемента зависят только от частоты тока и, следовательно, для всех последовательностей одинаковы\footnote{Такими элементами можно практически считать реакторы}, т. е.

\begin{equation*}
	r_1 = r_2 = r_0
\end{equation*}

и

\begin{equation*}
	x_1 = x_2 = x_0
\end{equation*}

соответственно

\begin{equation*}
	z_1 = z_2 = z_0
\end{equation*}

Для элемента, магнитносвязанные цепи которого неподвижны относительно друг друга, сопротивления прямой и обратной последовательностей одинаковы, так как от перемены порядка чередования фаз симметричной трехфазной системы токов взаимоиндукция между фазами такого элемента не изменяется.

Таким образом, для трансформаторов, автотрансформаторов, воздушных линий, кабелей и реакторов

\begin{equation*}
	r_1 = r_2
\end{equation*}

и

\begin{equation*}
	x_1 = x_2
\end{equation*}

соответственно

\begin{equation*}
	z_1 = z_2
\end{equation*}

Система токов нулевой последовательности резко отличается от систем токов прямой и обратной последовательностей, вследствие чего сопротивления нулевой последовательности в общем случае весьма существенно отличаются от соответствующих сопротивлений двух других последовательностей.

Помимо определения индуктивных сопротивлений обратной и нулевой последовательностей, ниже также приведены указания к определению активных сопротивлений нулевой последовательности воздушных и кабельных линий. Учет последних часто необходим при расчете однофазных коротких замыканий, причем его выполнение обычно не вызывает трудностей, так как этот вид короткого замыкания в большинстве случаев характеризуется большой электрической удаленностью, что позволяет не считаться с изменением тока во времени.

\section{Синхронные машины}
\label{sec:12-2}

Магнитный поток, созданный токами обратной последовательности синхронной частоты, вращаясь относительно ротора с двойной синхронной скоростью, встречает на своем пути непрерывно изменяющееся магнитное сопротивление; это обусловлено магнитной несимметрией ротора и тем, что наведенные в продольных и поперечных контурах ротора токи создают различные ответные реакции. Таким образом, при неизменной н.~с. статора поток обратной последовательности гармонически изменяется с двойной синхронной скоростью в пределах между его наибольшим и наименьшим значениями, разница между которыми зависит от степени несимметрии ротора; она велика при резкой несимметрии ротора и, напротив, совсем исчезает при его полной симметрии.

В §\ref{sec:11-2} было показано, что поток обратной последовательности синхронной частоты в общем случае вызывает в статоре нечетные гармоники, которые искажают синусоидальную форму магнитного поля статора. Это обстоятельство существенно затрудняет определение реактивности обратной последовательности синхронной машины и приводит к тому, что данная реактивность, строго говоря, не является параметром машины, так как она зависит от внешних условий (т.~е. внешней реактивности, вида несимметрии и др.).

Для синхронной машины без демпферных обмоток в §~\ref{sec:9-2} было получено выражение для реактивности

\begin{equation}
	x_2 = \frac{2x'_d x_q}{x'_d + x_q} \text{,}
	\label{eq:12-1}
\end{equation}

которая по существу представляет собой реактивность обратной последовательности, определяемую как отношение подведенного синусоидального напряжения обратной последовательности синхронной частоты к основной гармонике тока обратной последовательности.

Эта реактивность может быть представлена схемой замещения, показанной на рис.~\ref{fig:12-1}. Ток в параллельной ветви с реактивностью дает значение третьей гармоники тока прямой последовательности, которая вызвана потоком обратной последовательности синхронной частоты.

Представим себе теперь, что напряжение обратной последовательности приложено не непосредственно к статору машины, а через произвольную реактивность $ x $. Тогда общая реактивность обратной последовательности всей цепи, очевидно, будет:

\begin{equation*}
	x_{2\Sigma}  = \frac{2(x'_d + x) (x_q + x)}{x'_d + x_q + 2x}
\end{equation*}

и на долю самой машины приходится величина

\begin{equation*}
	x_{2}  = \frac{2(x'_d + x) (x_q + x)}{x'_d + x_q + 2x} - x = \frac{2x'_d x_q + (x'_d + x_q) x}{x'_d + x_q + 2x} \text{,}
\end{equation*}

которая, как видно, зависит от внешней реактивности $ x $. По мере увеличения последней реактивность обратной последовательности машины стремится в пределе к

\begin{equation}
	x_{2}  = \lim_{x \rightarrow \infty } = \frac{2x'_d x_q + (x'_d + x_q) x}{x'_d + x_q + 2x} = \frac{x'_d + x_q}{2} \text{,}
	\label{eq:12-2}
\end{equation}

что соответствует отсутствию третьей гармоники тока. Эта реактивность получается из схемы замещения рис.~\ref{fig:12-1}, для чего достаточно разомкнуть рубильник \textit{Р}.

Следовательно, принципиальная разница между выражениями (\ref{eq:12-1}) и (\ref{eq:12-2}) состоит в том, что первое из них дает значение $ х_2 $ машины с учетом влияния третьей гармоники тока, а второе -- без учета такого влияния. При симметричном роторе ($ х_q = х'_d $) оба выражения дают одно и тоже значение что также следует из схемы замещения рис.~\ref{fig:12-1}.

До сих пор предполагалось, что обратно-синхронное питание подано от источника бесконечной мощности, в силу чего, помимо основной гармоники, в статоре возникает еще только третья гармоника тока. Однако при несимметричном режиме машины (см.~§~\ref{sec:11-2}) поле обратной последовательности основной частоты вызывает в статоре весь спектр нечетных гармоник. В этом случае, как показал Н.~Н.~Щедрин, схема замещения рис.~\ref{fig:12-1} может быть развита в бесконечную цепную схему замещения, результирующая реактивность которой составляет:

\begin{equation}
	x_{2}  = \sqrt{x'_d x_q} \text{.}
	\label{eq:12-3}
\end{equation}

Эта реактивность также зависит от внешней реактивности и в пределе стремится к значению, определяемому по (\ref{eq:12-2}).

Для машины с демпферными обмотками реактивность $ x_2 $ может быть определена по тем же выражениям, если заменить в них $ x'_d $ и $ x_q $ соответственно $ x''_d $ и $ x''_q $. Величины реактивностей $ x''_d $ и $ x''_q $ обычно ближе друг к другу, чем величины $ x'_d $ и $ x_q $. Поэтому у машин с полным демпфированием разница в значениях $ x_2 $, получаемых по разным выражениям, очень мала.

Поскольку выражения (\ref{eq:12-1}) -- (\ref{eq:12-3}) почти равноценны, в большинстве практических расчетов целесообразно принимать для синхронных машин реактивность $ x_2 $ по наиболее простому выражению (\ref{eq:12-2}), которое к тому же удовлетворяет нормальному правилу последовательного соединения реактивностей машины и ее внешней цепи. При необходимости учета высших гармоник надлежит применять более точное выражение (\ref{eq:12-3}).

В качестве приближенных соотношений принимают:

%TODO: как выделить этот долбанный блок?
\begin{tabular}{lc}
	Для машин без демпферных обмоток & $ x_2 \approx 1,45 x'_d  $; \\ 
	Для турбогенераторов и машин с демпферными обмотками в обеих осях ротора &  $ x_2 \approx 1,22 x''_d  $. \\ 
\end{tabular} 

В практических приближенных расчетах обычно идут на дополнительное упрощение, принимая для турбогенераторов и машин с продольно-поперечными демпферными обмотками

\begin{equation}
	x_2 \approx x''_d \text{.}
	\label{eq:12-4}
\end{equation}

Токи нулевой последовательности создают практически только магнитные потоки рассеяния статорной обмотки, которые, как правило, меньше, чем при токах прямой или обратной последовательности, причем это уменьшение сильно зависит от типа обмотки. Поэтому величина $ x_0 $ синхронных машин колеблется в широких пределах:

\begin{equation}
	x_0 = (0,15 \div 0,6) x''_d \text{.}
	\label{eq:12-5}
\end{equation}

\section{Асинхронные двигатели}
\label{sec:12-3}


\section{Обобщенная нагрузка}
\label{sec:12-4}


\section{Трансформаторы}
\label{sec:12-5}


\section{Автотрансформаторы}
\label{sec:12-6}


\section{Воздушные линии}
\label{sec:12-7}


\section{Кабели}
\label{sec:12-8}