\chapter{Параметры элементов для токов обратной и нулевой последовательностей}
\label{chap:12}

\section{Общие замечания}
\label{sec:12-1}

Все сопротивления, которыми характеризуются отдельные элементы в нормальном симметричном режиме, а также в симметричном переходном процессе, по существу являются сопротивлениями прямой последовательности\footnote{Исключение составляет реактивность, используемая при определении постоянной времени $ T_a $ (см~§ \ref{chap:9-2})}. Этот термин вводить ранее не было нужды, поскольку токи были лишь одной последовательности.

При отсутствии магнитной связи между фазами какого-либо элемента его сопротивление не зависит от порядка чередования фаз тока. Активная и реактивная слагающие сопротивления такого элемента зависят только от частоты тока и, следовательно, для всех последовательностей одинаковы\footnote{Такими элементами можно практически считать реакторы}, т. е.

\begin{equation*}
	r_1 = r_2 = r_0
\end{equation*}

и

\begin{equation*}
	x_1 = x_2 = x_0
\end{equation*}

соответственно

\begin{equation*}
	z_1 = z_2 = z_0
\end{equation*}

Для элемента, магнитносвязанные цепи которого неподвижны относительно друг друга, сопротивления прямой и обратной последовательностей одинаковы, так как от перемены порядка чередования фаз симметричной трехфазной системы токов взаимоиндукция между фазами такого элемента не изменяется.

Таким образом, для трансформаторов, автотрансформаторов, воздушных линий, кабелей и реакторов

\begin{equation*}
	r_1 = r_2
\end{equation*}

и

\begin{equation*}
	x_1 = x_2
\end{equation*}

соответственно

\begin{equation*}
	z_1 = z_2
\end{equation*}

Система токов нулевой последовательности резко отличается от систем токов прямой и обратной последовательностей, вследствие чего сопротивления нулевой последовательности в общем случае весьма существенно отличаются от соответствующих сопротивлений двух других последовательностей.

Помимо определения индуктивных сопротивлений обратной и нулевой последовательностей, ниже также приведены указания к определению активных сопротивлений нулевой последовательности воздушных и кабельных линий. Учет последних часто необходим при расчете однофазных коротких замыканий, причем его выполнение обычно не вызывает трудностей, так как этот вид короткого замыкания в большинстве случаев характеризуется большой электрической удаленностью, что позволяет не считаться с изменением тока во времени.

\section{Синхронные машины}
\label{sec:12-2}

Магнитный поток, созданный токами обратной последовательности синхронной частоты, вращаясь относительно ротора с двойной синхронной скоростью, встречает на своем пути непрерывно изменяющееся магнитное сопротивление; это обусловлено магнитной несимметрией ротора и тем, что наведенные в продольных и поперечных контурах ротора токи создают различные ответные реакции. Таким образом, при неизменной н.~с. статора поток обратной последовательности гармонически изменяется с двойной синхронной скоростью в пределах между его наибольшим и наименьшим значениями, разница между которыми зависит от степени несимметрии ротора; она велика при резкой несимметрии ротора и, напротив, совсем исчезает при его полной симметрии.

В §\ref{sec:11-2} было показано, что поток обратной последовательности синхронной частоты в общем случае вызывает в статоре нечетные гармоники, которые искажают синусоидальную форму магнитного поля статора. Это обстоятельство существенно затрудняет определение реактивности обратной последовательности синхронной машины и приводит к тому, что данная реактивность, строго говоря, не является параметром машины, так как она зависит от внешних условий (т.~е. внешней реактивности, вида несимметрии и др.).

Для синхронной машины без демпферных обмоток в §~\ref{sec:9-2} было получено выражение для реактивности

\begin{equation}
	x_2 = \frac{2x'_d x_q}{x'_d + x_q} \text{,}
	\label{eq:12-1}
\end{equation}

которая по существу представляет собой реактивность обратной последовательности, определяемую как отношение подведенного синусоидального напряжения обратной последовательности синхронной частоты к основной гармонике тока обратной последовательности.

Эта реактивность может быть представлена схемой замещения, показанной на рис.~\ref{fig:12-1}. Ток в параллельной ветви с реактивностью дает значение третьей гармоники тока прямой последовательности, которая вызвана потоком обратной последовательности синхронной частоты.

Представим себе теперь, что напряжение обратной последовательности приложено не непосредственно к статору машины, а через произвольную реактивность $ x $. Тогда общая реактивность обратной последовательности всей цепи, очевидно, будет:

\begin{equation*}
	x_{2\Sigma}  = \frac{2(x'_d + x) (x_q + x)}{x'_d + x_q + 2x}
\end{equation*}

и на долю самой машины приходится величина

\begin{equation*}
	x_{2}  = \frac{2(x'_d + x) (x_q + x)}{x'_d + x_q + 2x} - x = \frac{2x'_d x_q + (x'_d + x_q) x}{x'_d + x_q + 2x} \text{,}
\end{equation*}

которая, как видно, зависит от внешней реактивности $ x $. По мере увеличения последней реактивность обратной последовательности машины стремится в пределе к

\begin{equation}
	x_{2}  = \lim_{x \rightarrow \infty } = \frac{2x'_d x_q + (x'_d + x_q) x}{x'_d + x_q + 2x} = \frac{x'_d + x_q}{2} \text{,}
	\label{eq:12-2}
\end{equation}

что соответствует отсутствию третьей гармоники тока. Эта реактивность получается из схемы замещения рис.~\ref{fig:12-1}, для чего достаточно разомкнуть рубильник \textit{Р}.

\begin{wrapfigure}{l}{0.5\linewidth} %\begin{wrapfigure}[высота в строках]]{r}{0.4\linewidth} % 1-1
	\centering
	\includesvg{pic/12-1}
	\caption{Схема замещения, определяющая реактивность $ x_2 $ синхронной машины с учетом влияния третьей гармоники тока прямой последовательности}
	\label{fig:12-1}
\end{wrapfigure}

Следовательно, принципиальная разница между выражениями (\ref{eq:12-1}) и (\ref{eq:12-2}) состоит в том, что первое из них дает значение $ х_2 $ машины с учетом влияния третьей гармоники тока, а второе -- без учета такого влияния. При симметричном роторе ($ х_q = х'_d $) оба выражения дают одно и тоже значение что также следует из схемы замещения рис.~\ref{fig:12-1}.

До сих пор предполагалось, что обратно-синхронное питание подано от источника бесконечной мощности, в силу чего, помимо основной гармоники, в статоре возникает еще только третья гармоника тока. Однако при несимметричном режиме машины (см.~§~\ref{sec:11-2}) поле обратной последовательности основной частоты вызывает в статоре весь спектр нечетных гармоник. В этом случае, как показал Н.~Н.~Щедрин, схема замещения рис.~\ref{fig:12-1} может быть развита в бесконечную цепную схему замещения, результирующая реактивность которой составляет:

\begin{equation}
	x_{2}  = \sqrt{x'_d x_q} \text{.}
	\label{eq:12-3}
\end{equation}

Эта реактивность также зависит от внешней реактивности и в пределе стремится к значению, определяемому по (\ref{eq:12-2}).

Для машины с демпферными обмотками реактивность $ x_2 $ может быть определена по тем же выражениям, если заменить в них $ x'_d $ и $ x_q $ соответственно $ x''_d $ и $ x''_q $. Величины реактивностей $ x''_d $ и $ x''_q $ обычно ближе друг к другу, чем величины $ x'_d $ и $ x_q $. Поэтому у машин с полным демпфированием разница в значениях $ x_2 $, получаемых по разным выражениям, очень мала.

Поскольку выражения (\ref{eq:12-1}) -- (\ref{eq:12-3}) почти равноценны, в большинстве практических расчетов целесообразно принимать для синхронных машин реактивность $ x_2 $ по наиболее простому выражению (\ref{eq:12-2}), которое к тому же удовлетворяет нормальному правилу последовательного соединения реактивностей машины и ее внешней цепи. При необходимости учета высших гармоник надлежит применять более точное выражение (\ref{eq:12-3}).

В качестве приближенных соотношений принимают:

%TODO: как выделить этот долбанный блок?
\begin{tabular}{lc}
	Для машин без демпферных обмоток & $ x_2 \approx 1,45 x'_d  $; \\ 
	Для турбогенераторов и машин с демпферными обмотками в обеих осях ротора &  $ x_2 \approx 1,22 x''_d  $. \\ 
\end{tabular} 

В практических приближенных расчетах обычно идут на дополнительное упрощение, принимая для турбогенераторов и машин с продольно-поперечными демпферными обмотками

\begin{equation}
	x_2 \approx x''_d \text{.}
	\label{eq:12-4}
\end{equation}

Токи нулевой последовательности создают практически только магнитные потоки рассеяния статорной обмотки, которые, как правило, меньше, чем при токах прямой или обратной последовательности, причем это уменьшение сильно зависит от типа обмотки. Поэтому величина $ x_0 $ синхронных машин колеблется в широких пределах:

\begin{equation}
	x_0 = (0,15 \div 0,6) x''_d \text{.}
	\label{eq:12-5}
\end{equation}

\section{Асинхронные двигатели}
\label{sec:12-3}

Если в нормальных условиях асинхронный двигатель работает со скольжением $ s $, то по отношению к магнитному потоку обратной последовательности синхронной частоты ротор двигателя, очевидно, имеет скольжение $ (2-s) $. Следовательно, сопротивление обратной последовательности асинхронного двигателя представляет собой его сопротивление при скольжении $ (2-s) $.

Кривая, показанная на рис.~\ref{fig:12-2}, иллюстрирует примерный характер относительного изменения реактивности асинхронного двигателя в функции скольжения\footnote{За единицу реактивности здесь принята реактивность двигателя при его номинальном скольжении.}.

Как видно, с ростом $ s $ реактивность двигателя вначале резко падает, а затем ее снижение весьма незначительно. Это позволяет практически считать

\begin{equation}
	x_2 \approx x_{s=1} = x_{\text{K}} \text{,}
	\label{eq:12-6}
\end{equation}

т.~е. реактивность $ x_2 $ двигателя равной его так называемой реактивности короткого замыкания (относительная величина которого близка к обратной величине относительного номинального пускового тока).

\begin{figure} %\begin{wrapfigure}[высота в строках]]{r}{0.4\linewidth} % 1-1
	\centering
	\includesvg{pic/12-2}
	\caption{Относительное изменение индуктивного сопротивления асинхронного двигателя в зависимости от скольжения.}
	\label{fig:12-2}
\end{figure}

Реактивность нулевой последовательности асинхронного двигателя, как и синхронных машин, определяется только рассеянием статорной обмотки и сильно зависит от типа и конструкции последней. Достаточно надежные значения этой реактивности могут быть получены преимущественно опытным путем или по данным завода-изготовителя.

\section{Обобщенная нагрузка}
\label{sec:12-4}

Реактивность обратной последовательности обобщенной нагрузки зависит от характера приемников электроэнергии и относительного участия каждого из них в рассматриваемой нагрузке. Для средней типовой промышленной нагрузки можно считать, что основная ее часть состоит из асинхронных двигателей, реактивность обратной последовательности которых, как показано в §~\ref{sec:12-3}, практически та же, что и в начальный момент внезапного нарушения режима. Поэтому для реактивности обратной последовательности обобщенной нагрузки в практических расчетах можно принимать, как и в §~\ref{sec:6-5}, величину

\begin{equation}
	x_2 = 0,35 \text{,}
	\label{eq:12-7}
\end{equation}

считая ее отнесенной к полной рабочей мощности в мегавольтамперах данной нагрузки и среднему номинальному напряжению той ступени, где она присоединена.

Поскольку обобщенная нагрузка включает в себя сеть и понижающие трансформаторы, ее сопротивление нулевой последовательности обычно определяется именно этими элементами, рассмотрение которых приведено ниже. Привести какие-либо средние величины этого сопротивления не представляется возможным.

% https://tex.stackexchange.com/questions/153329/footnote-in-sub-section-title
\section{Трансформаторы\protect\footnote{Для общности проводимых здесь записей обмотки трансформатора обозначены порядковыми номерами І, II, III вместо В, С, Н, как это обычно принято.}}
\label{sec:12-5}

Реактивность нулевой последовательности трансформатора в значительной мере определяется его конструкцией и соединением обмоток.

Со стороны обмотки, соединенной в треугольник или в звезду без заземленной нейтрали, независимо от того, как соединены другие обмотки, реактивность нулевой последовательности трансформатора, очевидно, бесконечно велика ($ x_0 = \infty $), так как при этих условиях вообще исключена возможность циркуляции тока нулевой последовательности в данном трансформаторе. Следовательно, конечная, хотя иногда (см.~ниже) и очень большая, реактивность нулевой последовательности трансформатора может быть только со стороны его обмотки, соединенной в звезду с заземленной нейтралью.

\begin{figure} %\begin{wrapfigure}[высота в строках]]{r}{0.4\linewidth} % 1-1
	%\centering
	\includesvg{pic/12-3}
	\caption{Соединения обмоток трансформаторов и их схемы замещения для токов нулевой последовательности.}
	\label{fig:12-3}
\end{figure}

На рис.~\ref{fig:12:3}, \textit{а}, \textit{б} и \textit{в} приведены основные варианты соединения обмоток двухобмоточного трансформатора, при которых приложенное к обмотке \textit{I} напряжение нулевой последовательности вызывает в одной или в обеих обмотках ток той же последовательности. Справа, против каждого варианта соединения обмоток показаны схемы замещения трансформатора (без учета активных сопротивлений) для токов нулевой последовательности.

При соединении обмоток $ \text{Y}_0 / \Delta $ (рис.~\ref{fig:12-3},~\textit{а}) э.~д.~с. нулевой последовательности трансформатора целиком расходуется на проведение тока той же последовательности только через реактивность рассеяния обмотки, соединенной треугольником, так как этот ток (подобно третьей гормонике тока) не выходит за пределы данной обмотки. В схеме замещения это отражают закорачиванием ветви с $ x_{II} $. Потенциал, равный нулю, на конце ветви $ x_{II} $ схемы замещения не указывает на искусственный перенос заземления нейтрали, как это иногда ошибочно воспринимают; он только соответствует условию, что данной ветвью схемы замещения трансформатора заканчивается путь циркуляции токов нулевой последовательности.

При соединении обмоток $ \text{Y}_0 / \text{Y}_0 $ представленная на рис.~\ref{fig:12-3},~\textit{б} схема замещения предполагает, что на стороне обмотки \textit{II} обеспечен путь для тока нулевой последовательности, т.~е. в цепи этой обмотки имеется по меньшей мере еще одна заземленная нейтраль (см.~пунктир). Если же этого нет, то схема замещения будет такой же, как и при соединении обмоток $ \text{Y}_0 / \text{Y} $ (рис.~\ref{fig:12-3},~\textit{в}), что соответствует режиму холостого хода трансформатора.

Оценим теперь величину реактивности намагничивания нулевой последовательности трансформатора $ x_{\mu_0} $.

Для группы из трех однофазных трансформаторов, а также для трехфазных четырех- и пятистержневых (броневых) трансформаторов ток намагничивания нулевой последовательности очень мал, так как в этом случае условия для магнитного потока практически те же, что и при питании трансформатора от источника напряжения прямой (или обратной) последовательности. Поэтому в соответствии с принятым ранее (§~\ref{sec:2-1}) допущением можно считать $ x_{\mu_0} = \infty $.

Иные условия имеют место в трехфазных трехстержневых трансформаторах, где магнитные потоки нулевой последовательности вынуждены замыкаться через изолирующую среду и кожух трансформатора. Для проведения магнитного потока по пути со столь высоким магнитным сопротивлением необходим достаточно большой ток намагничивания; следовательно, реактивность $ x_{\mu_0} $ у трансформатора такого типа значительно меньше, чем $ x_{\mu_1} $. В зависимости от конструкции этого типа трансформатора она находится в пределах $ x_{\mu_0} = (0,3 \div 1,0 ) $. Имея в виду, что величина $ x_{II} $ все же значительно меньше $ x_{\mu_0} $, можно практически считать, что и для трехстержневого трансформатора с соединением обмоток $ \text{Y}_0 / \Delta $ ~~ $ x_{\mu_0} \approx \infty $.

В табл.~\ref{tabl:12-1} сведены изложенные выше указания относительно оценки реактивности нулевой последовательности двухобмоточных трансформаторов.

%TODO: Таблица 12-1

У трехобмоточных трансформаторов одна из обмоток, как правило, соединена в треугольник. Поэтому для них всегда можно принимать $ x_{\mu_0} = \infty $.

Основные варианты соединения обмоток трехобмоточного трансформатора и соответствующие им схемы замещения нулевой последовательности (считая $ U_0 $ приложенным со стороны обмотки \textit{I}) приведены на рис.~\ref{fig:12-3},~\textit{г}, \textit{д} и \textit{e}.

В варианте рис.~\ref{fig:12-3},~\textit{г} ток нулевой последовательности в обмотке \textit{III} отсутствует. Следовательно, в этом случае $ x_0 = x_I + x_{II} + x_{I-II} $.

В варианте рис.~\ref{fig:12-3},~\textit{д} предполагается, что путь для тока нулевой последовательности на стороне обмотки \textit{III} обеспечен. В этом случае в схему нулевой последовательности трансформатор должен быть введен своей схемой замещения.

Наконец, в варианте ~\ref{fig:12-3},~\textit{e} компенсация тока нулевой последовательности обмотки \textit{I} осуществляется токами, наведенными в обмотках \textit{II} и \textit{III}. В этом случае

\begin{equation*}
	x_0 = x_1 + \frac{ x_{II} x_{III} }{ x_{II} + x_{III} } \text{.}
\end{equation*}

\section{Автотрансформаторы\protect\footnote{См.~сноску к §~\ref{sec:12-5}}}
\label{sec:12-6}

Обмотки автотрансформатора связаны между собой не только магнитно, но и электрически; поэтому здесь иные условия для протекания токов нулевой последовательности, которые должны быть отражены в схеме замещения нулевой последовательности автотрансформатора. При известных условиях, как показано ниже, даже при изолированной нейтрали автотрансформатора в его обмотках возможна циркуляция токов нулевой последовательности.

При глухом заземлении нейтрали автотрансформатора его схема замещения нулевой последовательности аналогична схеме соответствующего трансформатора. Так, если у автотрансформатора нет третьей обмотки и во вторичной цепи обеспечен путь для тока нулевой последовательности, его схема замещения (при пренебрежении намагничивающим током и активными сопротивлениями) представляется суммарной реактивностью рассеяния (рис.~\ref{fig:12-4},~\textit{а}). При наличии третьей обмотки\footnote{Силовые автотрансформаторы, как правило, снабжены такой обмоткой}, соединенной треугольником, схема замещения имеет тот же вид, что и у трехобмоточного трансформатора при соответственном соединении его обмоток (рис.~\ref{fig:12-4},~\textit{в}).

Следует подчеркнуть, что непосредственно из схемы замещения нулевой последовательности автотрансформатора нельзя получить ток, протекающий в его нейтрали. При указанных на рис.~\ref{fig:12-4} направлениях токов искомый ток в нейтрали равен утроенной разности токов нулевой последовательности первичной и вторичной цепей, т.~е. $ \overset{~\cdot}{I}_N = 3( \overset{~\cdot}{I}_{0I} - \overset{~\cdot}{I}_{0II}) $, причем каждый из них должен быть отнесен к своей ступени напряжения, а не к какой-либо одной, для которой составлена схема замещения.

%TODO: Рисунок 12-4

Допустим теперь, что нейтраль автотрансформатора заземлена через реактивность $ X_N $ (рис.~\ref{fig:12-4},~\textit{б}). Если напряжение на нейтрали равно $ U_N $ и напряжения выводов ступеней \textit{I} и \textit{II} относительно нейтрали составляют соответственно $ U_{N\text{I}} $ и $ U_{N\text{II}} $ то для результирующей реактивности нулевой последовательности между выводами ступеней \textit{I} и \textit{II} автотрансформатора, приведенной к ступени \textit{I}, можно написать:

\begin{equation*}
	x'_{I-II} = \frac{( U_{N\text{I}} + U_N ) - ( U_{N\text{II}} + U_N ) U_{\text{I}} / U_{\text{II}} }{I_{0\text{I}}} = \frac{ U_{N\text{I}} - \overset{\circ }{U}_{N\text{II}} }{ I_{0\text{I}} } + \frac{U_N}{I_{0\text{I}}}\left ( 1- \frac{U_{\text{I}}}{U_{\text{II}}} \right ) \text{;}
\end{equation*}

поскольку

\begin{equation*}
	\frac{ U_{N\text{I}} - \overset{\circ }{U}_{N\text{II}} }{ I_{0\text{I}} } = x_{\text{I-II}} \text{,}
\end{equation*}

где $ x_{\text{I-II}} $ -- реактивность рассеяния автотрансформатора, отнесенная к ступени \textit{I},

и

\begin{equation*}
	\frac{ U_{N} }{ I_{0\text{I}} } = \frac{ 3x_N ( I_{0\text{I}} - I_{0\text{II}} ) }{ I_{0\text{I}} } = 3x_N \left ( 1- \frac{U_{\text{I}}}{U_{\text{II}}} \right ) \text{,}
\end{equation*}

то окончательно получим:

\begin{equation}
	x'_{\text{I-II}} = x_{\text{I-II}} + 3x_N \left ( 1- \frac{U_{\text{I}}}{U_{\text{II}}} \right )^2 \text{.}
	\label{eq:12-8}
\end{equation}

Данное выражение, разумеется, справедливо также и в том случае, когда реактивности представлены в относительных единицах, причем его запись предполагает, что реактивность $ x_N $ отнесена к базисному напряжению ступени \textit{I}.

Аналогичным образом для автотрансформатора, имеющего третью обмотку, соединенную треугольником (рис.~\ref{fig:12-4},~\textit{г}), нетрудно найти результирующие реактивности нулевой последовательности между другими парами его обмоток, также отнесенные к ступени \textit{I}:

\begin{equation}
	x'_{\text{I-III}} = x_{\text{I-III}} + 3x_N \text{;}
	\label{eq:12-9}
\end{equation}

\begin{equation}
	x'_{\text{I-III}} = x_{\text{II-III}} + 3x_N \left (\frac{U_{\text{I}}}{U_{\text{II}}} \right )^2 \text{.}
	\label{eq:12-10}
\end{equation}

Используя (\ref{eq:12-8}) -- (\ref{eq:12-10}), по известным формулам для трехобмоточного трансформатора (см.~приложение \colorbox{red}{П-7}) находим реактивности трехлучевой схемы замещения:

% промежутки между строк матрицы - https://tex.stackexchange.com/questions/14071
\begin{equation}
	\left.\begin{matrix}
	x'_{\text{I}} = x_{\text{I}} + 3x_N \left ( 1- \frac{U_{\text{I}}}{U_{\text{II}}} \right ) \text{;} \\[1em]
	x'_{\text{II}} = x_{\text{II}} + 3x_N \frac{(U_{\text{I}} - U_{\text{II}})U_{\text{I}}}{U_{\text{II}}^2} \text{;} \\[1em]
	x'_{\text{III}} = x_{\text{III}} + 3x_N \frac{U_{\text{I}}}{U_{\text{II}}} \text{.}
	\end{matrix}\right\}
	\label{eq:12-11}
\end{equation}

У автотрансформатора без третьей обмотки разземление нейтрали приводит к тому, что в схеме нулевой последовательности такой автотрансформатор оказывается в режиме холостого хода; его ток намагничивания достаточно мал и им можно пренебречь, поэтому $ x_{\mu_0} = \infty $.

Иные условия имеют место при разземлении нейтрали автотрансформатора, который снабжен третьей обмоткой, соединенной треугольником. В этом случае циркуляция тока нулевой последовательности возможна \cite{04Ulianov64}.

\section{Воздушные линии}
\label{sec:12-7}

Ток нулевой последовательности воздушной линии возвращается через землю и по заземленным цепям, расположенным параллельно данной линии (защитные тросы, рельсовые пути вдоль линии и пр.). Главная трудность достоверного определения сопротивления нулевой последовательности воздушной линии связана с учетом распределения тока в земле; Точное нахождение последнего в общем виде представляет собой весьма сложную проблему. Достаточно полное и строгое решение в предположении постоянства электрической проводимости земли и неограниченности ее размеров выполнено Карсоном. Установленные на основании его выводов приближенные формулы позволяют с достаточной для практики точностью вычислить отдельные составляющие и полное сопротивление нулевой последовательности воздушной линии при токах промышленной частоты и обычно встречающихся значениях проводимости земли. Эти формулы с краткими пояснениями приведены ниже, причем их окончательный вид дан для частоты $ f = 50~\text{гц} $.

Распределение переменного тока в земле выражается сложной закономерностью, аналогичной закономерности распределения тока в массивных проводниках.

Представим себе однопроводную линию переменного тока, обратным проводом которой служит земля (рис.~\ref{fig:12-5}), или, как ее иначе называют, линию «провод -- земля». Характер изменения плотности тока в земле по мере удаления в стороны и углубления в землю иллюстрируют кривые, показанные на рис.~\ref{fig:12-5}. Ток в земле как бы подтягивается к проводнику; соответственно наибольшая плотность тока имеет место на поверхности земли непосредственно под самим проводником.

Индуктивность такой линии, как показал Карсон, может быть определена как индуктивность эквивалентной двухпроводной линии с расстоянием между проводами $ D_{\text{з}} $ (рис.~\ref{fig:12-5}). Это расстояние называется \so{эквивалентной глубиной возврата тока через землю} и может быть определено по формуле, предложенной Карсоном:

\begin{equation}
	D_{\text{з}} = \frac{2,085}{\sqrt{f \lambda 10^{-9}}} \cdot  10^{-3} \text{, \textit{м},}
	\label{eq:12-12}
\end{equation}

где $ f $ -- частота тока, \textit{гц};
$ \lambda $ -- удельная проводимость земли, $ 1/\textit{ом} \cdot \textit{см} $.

При $ f = 50 \textit{гц} $ и среднем значении $ \lambda = 10^{-4} 1/\textit{ом} \cdot \textit{см} $ величина $ D_{\textit{з}} = 935 \textit{м} $. При отсутствии данных о проводимости земли обычно принимают $ D_{\text{з}} = 1000~\textit{м} $.

Таким образом, индуктивное сопротивление линии «провод -- земля» легко определить по известной формуле для двухпроводной линии:

\begin{equation}
	x_L = 0,145 lg \frac{D_{\text{з}}}{r_{\text{э}}} \text{, \textit{ом/км}.}
	\label{eq:12-13}
\end{equation}

Здесь $ r_{\text{э}} $ -- эквивалентный радиус провода, значения которого составляют:

\begin{small} % пример 2-1	
	\vspace{1pc}   		
	\begin{tabular}{p{0.70\linewidth}p{0.20\linewidth}}
		\parbox{1\linewidth}{Для витых медных проводников в зависимости от числа прядей} & $ r_{\text{э}} = (0,724\div0,771) r $; \\ 
		\multirow{2}{1\linewidth}{Для алюминиевых и сталеалюминиевых проводов марки А, АС, АСО, АСУ} & $ r_{\text{э}} = 0,95 r $; \\
		& где $ r $ -- истинный радиус провода. \\ 
	\end{tabular} 
	\vspace{1pc}
\end{small}

Для линий с расщепленными проводами в (\ref{eq:12-13}) вместо $ r_{\text{э}} $ следует вводить средний геометрический радиус $ r_{\text{ср}} $ системы проводов одной фазы, который находят из выражения:

\begin{equation}
	r_{\text{ср}} = \sqrt[n]{r_{\text{э}} a_{\text{ср}}^{n-1}} \text{,}
	\label{eq:12-14}
\end{equation}

где $ n $ -- число проводов в фазе;\\
$ a_{\text{ср}} $ -- среднее геометрическое расстояние между проводами одной фазы.

Все линейные величины, входящие под знак логарифма в (\ref{eq:12-13}) и дальнейших формулах, должны быть выражены в одних и тех же (вообще говоря, произвольных) единицах.

Активное сопротивление линии «провод -- земля» складывается из активного сопротивления провода $ r_{\text{п}} $ и дополнительного сопротивления $ r_{\text{з}} $, учитывающего потерю активной мощности в земле от протекающего в ней тока, т.~е.

\begin{equation}
	r = r_{\text{п}} + r_{\text{з}} \text{.}
	\label{eq:12-15}
\end{equation}

Сопротивление $ r_{\text{з}} $ может быть определено из приближенного выражения:

\begin{equation}
	r_{\text{з}} = \pi f \cdot 10^{-4} r_{\text{з}} \text{, \textit{ом/км} ,}
	\label{eq:12-16}
\end{equation}

которое при $ f = 50 \textit{гц} $ дает $ r_{\text{з}} = 0,05 \textit{ом/км} $.

Следует обратить внимание, что величина $ r_{\text{з}} $ практически не зависит от проводимости земли. Этот на первый взгляд парадоксальный вывод объясняется тем, что с изменением проводимости земли плотность тока в ней меняется; при этом потери активной мощности при заданной частоте тока остаются почти постоянными. Линейная зависимость $ r_{\text{з}} $ от частоты $ f $ вызвана соответственным проявлением поверхностного эффекта земли.

Сопротивление, обусловленное взаимоиндукцией между двумя параллельными линиями «провод -- земля» с расстоянием $ d $ между осями их проводов (если оно значительно меньше величины $ D_{\text{з}} $), можно определить из выражения:

\begin{equation}
	Z_M = 0,05 + j0,145 lg \frac{D_{\text{з}}}{d} \text{, \textit{ом/км} ,}
	\label{eq:12-17}
\end{equation}

где активная составляющая соответствует потере активной мощности, возникающей от протекания тока в земле.

Для трехфазной однодепной линии с полным циклом транспозиции проводов сопротивление взаимоиндукции между фазами при возврате тока через землю следует определять по (\ref{eq:12-17}), заменив в последнем $ d $ \so{средним геометрическим расстоянием между проводами фаз} \textit{a}, \textit{b} и \textit{c}:

\begin{equation}
	D_{\text{ср}} = \sqrt[3]{d_{ab} d_{ac} d_{bc}}
	\label{eq:12-18}
\end{equation}

т.~е.

\begin{equation}
	Z_{M\text{ср}} = 0,05 + j0,145 lg \frac{D_{\text{з}}}{D_{\text{ср}}} \text{, \textit{ом/км} ,}
	\label{eq:12-19}
\end{equation}

Зная $ Z_L $ и $ Z_{M\text{ср}} $, нетрудно найти сопротивление нулевой последовательности одноцепной трехфазной линии. Оно численно равно эффективному значению напряжения, которое должно быть приложено к каждому проводу данной линии, чтобы покрыть падение напряжения при протекании в фазах токов нулевой последовательности с эффективным значением $ 1 a $ ($ \overset{~\cdot}{I}_a = \overset{~\cdot}{I}_b = \overset{~\cdot}{I}_c = 1 a $), т.~е.

%TODO: Опечатка чтоли? откуда взялось L_L ?
\begin{equation}
	Z_0 = Z_L + Z_{M\text{ср}} + Z_{M\text{ср}} = L_L + 2 Z_{M\text{ср}} \text{;}
	\label{eq:12-20}
\end{equation}

после подстановки (\ref{eq:12-13}), (\ref{eq:12-15}), (\ref{eq:12-16}) и (\ref{eq:12-19}) и преобразования имеем:

\begin{equation}
	Z_0 = r_{\text{п}} + 0,15 + j0,435 lg \frac{D_{\text{з}}}{R_{\text{ср}}} \text{, \textit{ом/км} ,}
	\label{eq:12-21}
\end{equation}

где $ R_{\text{ср}} = \sqrt[3]{r_{\text{з}} D_{\text{ср}}^2} $ -- средний геометрический радиус системы трех проводов линии.

Здесь уместно заметить, что если по линии протекает ток прямой последовательности с эффективным значением $ 1 a $ (т.~е. $ \overset{~\cdot}{I}_a = 1, ~\overset{~\cdot}{I}_b = a^2, ~\overset{~\cdot}{I}_c = a $), то\footnote{Возможность представления трехфазной линии при протекании по ней токов прямой (или обратной) последовательности тремя линиями «провод -- земля» вытекает из того, что результирующее влияние фиктивных обратных проводов этих линий практически отсутствует, так как сумма токов в этих проводах равна нулю.}

\begin{equation}
	Z_1 = Z_L + a^2 Z_{M\text{ср}} + a Z_{M\text{ср}} = Z_L - Z_{M\text{ср}} \text{;}
	\label{eq:12-22}
\end{equation}

после подстановки и преобразования имеем известную формулу:

\begin{equation}
	Z_1 = r_{\text{п}} + 0.145 lg \frac{D_{\text{ср}}}{r_{\text{э}}}  \text{, \textit{ом/км} .}
	\label{eq:12-23}
\end{equation}

Из (\ref{eq:12-20}) и (\ref{eq:12-22}) непосредственно следует, чем вызвано различие между $ Z_1 $ и $ Z_0 $ линии. В то время как при токе прямой (или обратной) последовательности взаимоиндукция с другими фазами уменьшает сопротивление фазы, при токе нулевой последовательности она, напротив, увеличивает его. Поскольку $ Z_L $ и $ Z_{M\text{ср}} $ у линий соизмеримы, величины $ Z_1 $ и $ Z_0 $ резко отличаются между собой.

Из тех же выражений вытекают важные соотношения:

\begin{equation}
	Z_L = (Z_0 + 2Z_1) / 3  \text{;}
	\label{eq:12-24}
\end{equation}

\begin{equation}
	Z_{M\text{ср}} = (Z_0 - Z_1) / 3  \text{.}
	\label{eq:12-25}
\end{equation}

Сопротивление нулевой последовательности каждой цепи двухцепной линии дополнительно увеличивается\footnote{При прохождении токов в обеих цепях в одну сторону.} благодаря взаимоиндукции с проводами параллельной цепи. Сопротивление взаимоиндукции между проводом одной цепи и тремя проводами другой цепи можно определить по (\ref{eq:12-19}), где обе составляющие должны быть увеличены в 3 раза и вместо $ D_{\text{ср}} $ введено \so{среднее геометрическое расстояние $ D_{I-II} $ между цепями}, определяемое через расстояния от каждого провода (\textit{a}, \textit{b}, \textit{c}) цепи \textit{I} до каждого провода (\textit{a'}, \textit{b'}, \textit{c'}) цепи \textit{II}:

\begin{equation}
	D_{I-II} = \sqrt[9]{d_{aa'} d_{ab'} d_{ac'} d_{ba'} d_{bb'} d_{bc'} d_{ca'} d_{cb'} d_{cc'} }  \text{.}
	\label{eq:12-26}
\end{equation}

Следовательно, искомое сопротивление

\begin{equation}
	Z_{I-II0} = 0,15 + 0.435 lg \frac{D_{\text{з}}}{D_{I-II} }  \text{, \textit{ом/км} .}
	\label{eq:12-27}
\end{equation}

Следовательно, сопротивление нулевой последовательности двухцепной линии, считая в общем случае цепи неодинаковыми, определяют из известной схемы замещения двух магнитносвязанных цепей, как показано на рис.~\ref{fig:12-6}. При идентичности параллельных цепей ($ Z_{I0} = Z_{II0} = Z_0 $) сопротивление нулевой последовательности каждой из них, очевидно, будет:

\begin{equation}
	Z'_0 = Z_0 + Z_{I-II0}
	\label{eq:12-28}
\end{equation}

и их результирующее сопротивление (т.~е. сопротивление на фазу двухцепной линии)

\begin{equation}
	Z''_0 = 0,5 Z'_0 = 0,5 ( Z_0 + Z_{I-II0} )  \text{.}
	\label{eq:12-29}
\end{equation}

Отметим, что увеличение сопротивления нулевой последовательности вследствие влияния взаимоиндукции параллельной цепи не превышает 10\% при расстояниях между цепями порядка 400--500~\textit{м}. Разумеется, при расположении обеих цепей на одной опоре взаимоиндукция проявляется особенно сильно.

Оценим теперь влияние тросов на величину сопротивления нулевой последовательности линии. Тросы используют в качестве средства грозозащиты линии, располагая их в верхней точке опоры. До последнего времени тросы заземляли практически на каждой опоре. В образующихся при этом короткозамкнутых контурах возможно протекание наведенных токов. Последние малы при протекании по линии уравновешенной системы токов, в силу чего $ X_1 $ (и $ Z_2 $) практически не зависит от наличия заземленных тросов, и, напротив, могут быть значительны при протекании по линии тока нулевой последовательности. При этом ответная реакция от наведенных токов в тросе может существенно изменить сопротивление нулевой последовательности линии.

В последнее время для линий (преимущественно большой протяженности) стали применять другую систему заземления защитных тросов. Тросы подвешивают на изоляторах и разрезают на ряд участков. С одного конца каждого участка тросы заземляют, а с другого между тросом и землей оставляют искровой промежуток, пробой которого наступает при возникновении перенапряжения определенной величины. При такой системе заземления тросов последние практически не сказываются на сопротивлении нулевой последовательности линии.

Однако, поскольку у большинства существующих линий передачи защитные тросы заземлены почти на каждой опоре, определение сопротивления нулевой последовательности таких линий по-прежнему представляет практический интерес.

Обратимся к принципиальной схеме рис.~\ref{fig:12-7},~\textit{а}, где показаны одноцепная трехфазная линия с одним заземленным тросом и путь циркуляции токов нулевой последовательности для рассматриваемого случая.

Сумма токов в тросе $ \overset{~\cdot}{I}_T $ и земле $ \overset{~\cdot}{I}_{\text{З}} $ образует ток в нейтрали, равный утроенному току нулевой последовательности в фазе линии. Соответственно этому токи нулевой последовательности в тросе и земле, очевидно, будут:

\begin{equation*}
	\overset{~\cdot}{I}_{T0} = \overset{~\cdot}{I}_{T} / 3
\end{equation*}

и

\begin{equation*}
	\overset{~\cdot}{I}_{\text{З}0} = \overset{~\cdot}{I}_{\text{З}} / 3 \text{.}
\end{equation*}

Рассматривая трос как независимую однопроводную линию «провод--земля», его составляющие сопротивления можно определить по (\ref{eq:12-13}) и (\ref{eq:12-15}) и затем увеличить в 3 раза, чтобы учесть влияние токов всех фаз линии, что при $ f = 50~ \text{гц} $ дает:

\begin{equation}
	Z_{T0} = 3 r_T + 0,15 + 0.435 lg \frac{D_{\text{з}}}{r_{\textit{эT}}}  \text{, \textit{ом/км} ,}
	\label{eq:12-30}
\end{equation}

где $ r_T $ -- активное сопротивление троса;\\
$ r_{\textit{эT}} $ — эквивалентный радиус троса.

Сопротивление взаимоиндукции нулевой последовательности между проводом линии и тросом определяется по (\ref{eq:12-27}), где нужно заменить $ D_{I-II} $ средним геометрическим расстоянием между проводами и тросом (рис.~\ref{fig:12-8}), т.~е.

\begin{equation*}
	D_{\text{ПТ}} = \sqrt[3]{d_{a\text{Т}} d_{b\text{Т}} d_{c\text{Т}} } \text{.}
\end{equation*}

Зная $ Z_0 $, $ Z_{\text{Т}0} $ и $ Z_{\text{ПТ0}} $, нетрудно от схемы с магнитной
связью между цепями (рис.~\ref{fig:12-7},~\textit{б}) перейти к известной схеме замещения (рис.~\ref{fig:12-7},~\textit{в}), результирующее сопротивление которой дает искомое сопротивление нулевой последовательности линии с учетом заземленных тросов, т.~е.

\begin{equation}
	 Z_{0}^{(\text{Т})} = Z_0 - \frac{ Z_{\text{ПТ0}}^2 }{ Z_{\text{Т}0} }  \text{.}
	\label{eq:12-31}
\end{equation}

Чтобы нагляднее представить себе влияние заземленных тросов на сопротивление нулевой последовательности линии, запишем в последнем выражении сопротивления $ Z_0 $, $ Z_{\text{Т}0} $ и $ Z_{\text{ПТ0}} $ в показательной форме с аргументами соответственно $ \phi_0 $, $ \phi_{\text{Т}0} $, $ \phi_{\text{ПТ}0} $, т.~е.

\begin{equation}
	\label{eq:12-31a}	
	Z_{0}^{(\text{Т})} = z_0 e^{j\phi_0} - \frac{ z_{\text{ПТ0}}^2 }{ z_{\text{Т}0} } e^{j\psi} \text{,}
	\tag{\ref*{eq:12-31}а}
\end{equation}

где

\begin{equation*}
	\psi = (2 \phi_{\text{ПТ0}} - \phi_{\text{Т0}}) \text{,}
\end{equation*}

и на основании (\ref{eq:12-31a}) построим векторную диаграмму (рис.~\ref{fig:12-9}).

Значения указанных аргументов обычно находятся в следующих пределах:

\begin{small}	
	\vspace{1pc}   		
	\begin{tabular}{p{0.60\linewidth}p{0.30\linewidth}}
		при проводниках из цветных металлов & $ \phi = 75 \div 80^\circ $; \\ 
		при стальных проводниках & $ \phi = 30 \div 45^\circ $; \\ 
		при расстоянии $ D_{\text{ПТ0}} \leq 100 \textit{м} $ & $ \phi = 70 \div 80^\circ $.\\
	\end{tabular} 
	\vspace{1pc}
\end{small}

При этом, как видно из рис.~\ref{eq:12-9}, тросы всегда снижают индуктивное сопротивление нулевой последовательности линии (это также следует из самой роли тросов). Что касается их влияния на активное сопротивление нулевой последовательности линии, то здесь нельзя дать однозначный ответ. Как следует из рис.~\ref{fig:12-9}, при $ \psi > 90^\circ $, что имеет место при тросах с большим активным сопротивлением, сопротивление $ r_0 $ возрастает; при $ \psi < 90^\circ $, что имеет место при тросах с малым активным сопротивлением, $ r_0 $, напротив, снижается; в частном случае, при  $ \psi = 90^\circ $ оно остается без изменения.

Величины токов в тросе и земле легко определить, используя схему замещения рис.~\ref{fig:12-7},~\textit{в}:

\begin{equation}
	\overset{~\cdot}{I}_{\text{T}} = 3 \overset{~\cdot}{I}_{\text{T}0} = 3 \overset{~\cdot}{I}_0 \frac{Z_{\text{ПТ0}}}{Z_{\text{Т0}}}
	\label{eq:12-32}
\end{equation}

и

\begin{equation}
	\overset{~\cdot}{I}_{\text{З}} = 3 \overset{~\cdot}{I}_{\text{З}0} = 3 \overset{~\cdot}{I}_0 (I - \frac{Z_{\text{ПТ0}}}{Z_{\text{Т0}}} ) \text{.}
	\label{eq:12-33}
\end{equation}

При нескольких заземленных тросах их влияние на сопротивление нулевой последовательности сказывается, разумеется, сильнее. Аналогичные условия имеют место, когда одна из параллельных линий выведена в ремонт и заземлена. Методика подсчета параметров в подобных случаях указана в \cite{03Shchedrin55}, \cite{05Ulianov64}.

В приближенных практических расчетах в качестве средних соотношений между индуктивными сопротивлениями $ x_0 $ И $ x_1 $ для воздушных линий можно принимать значения, приведенные в табл.~\ref{tabl:12-2}.

%TODO здесь сноска должна быть со звездочкой, не нашел как это легко сделать
Реактивное емкостное сопротивление воздушной линии (на \textit{l~км}\footnote{Напомним, что емкостное сопротивление линии является перечным. Поэтому такое сопротивление линии длиной \textit{l~км} меньше сопротивления ее \textit{1~км} в \textit{l} раз.}) определяется по следующим выражениям \cite{02Shchedrin35}, \cite{09Vagner36}:










\section{Кабели}
\label{sec:12-8}
































