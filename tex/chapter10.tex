\chapter{Практические методы расчета переходного процесса короткого замыкания}
\label{chap:10}

\section{Общие замечания}
\label{sec:10-1}

Полученные в гл.~\ref{chap:9} общие выражения для тока при внезапном коротком замыкании позволяют с высокой точностью определить его величину в произвольный момент переходного процесса в цепи, питаемой одним генератором. Структура этих выражений показывает, что даже при столь простых условиях их применение требует большой вычислительной работы.

При переходе к схемам с несколькими генераторами, как показано в §~\ref{sec:9-6}, задача точного расчета переходного процесса короткого замыкания резко усложняется. Оставляя в стороне вопросы учета возникающих качаний генераторов и поведения присоединенных нагрузок, достаточно вспомнить, что изменения свободных токов в каждом из генераторов взаимно связаны между собой. При автоматическом регулировании возбуждения аналогичная связь имеет место также в приращениях принужденных токов. Трудность точного расчета дополнительно усугубляется различием параметров синхронной машины в продольной и поперечной осях ее ротора.

Использование приемов операционного исчисления для расчета переходных процессов короткого замыкания в мало-мальски сложной схеме сопряжено с преодолением весьма громоздких и трудоемких выкладок. Порядок характеристического уравнения быстро возрастает с увеличением числа машин в рассматриваемой схеме. Поэтому практическое применение такого метода расчета весьма ограничено. Его можно рассматривать лишь как эталон для оценки других приближенных методов расчета.

В силу указанных причин и с учетом того, что для решения многих практических задач не требуется знания точных результатов, разработаны приближенные методы расчета переходного процесса короткого замыкания. В дальнейшем рассмотрены только те из них, которые достаточно широко используются главным образом в практике советской электроэнергетики.

Основное требование, которому должен удовлетворять практический метод, заключается в простоте его выполнения, что прежде всего предотвращает возможность ошибок. Однако чем проще метод, тем на большем числе допущений он основан и тем, очевидно, меньше его точность. Самые простые методы позволяют иногда определить лишь порядок искомых величин, но этого часто бывает достаточно, чтобы обоснованно решить некоторые практические задачи. Почти, как правило, можно рекомендовать начать расчет переходного процесса короткого замыкания самым простым методом, а затем, если это требуется, вводить уточнения.

Помимо ранее указанных допущений (см.§~\ref{sec:2-1}), в практических расчетах коротких замыканий дополнительно принимают, что:

% остановился 2017-08-29

\section{Приближенный учет системы}
\label{sec:10-2}


\section{Расчет для выбора выключателей по отключающей способности}
\label{sec:10-3}


\section{Метод расчетных кривых}
\label{sec:10-4}


\section{Уточнение метода расчетных кривых}
\label{sec:10-5}


\section{Метод спрямленных характеристик}
\label{sec:10-6}