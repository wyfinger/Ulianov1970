\chapter{Уравнения электромагнитного переходного процесса синхронной машины}
\label{chap:7}

\section{Общие замечания и допущения}
\label{sec:7-1}

Ранее уже отмечалось, что аналитическое исследование электромагнитного переходного процесса синхронной машины с учетом всех влияющих на него факторов представляет чрезвычайно сложную задачу. Чтобы несколько упростить ее, приходится вводить ряд допущений, придавая машине некоторые свойства и качества, которыми она в действительности не обладает, т.~е. рассматривать в известной мере «идеализированную» машину. Несомненно, это вносит погрешности в оценку отдельных величин, однако, как показывает сопоставление получаемых величин с экспериментальными данными, обычно погрешности находятся в практически допустимых пределах. Следует особо подчеркнуть, что возможность использования тех или иных конкретных допущений зависит главным образом от характера и назначения решаемой задачи.

В §~\ref{sec:2-1}{2-1} были изложены основные допущения, обычно принимаемые в практических расчетах электромагнитных переходных процессов. Представляется полезным повторить некоторые из них и отметить часть дополнительных допущений, которые используются в дальнейшем. К таким допущениям нужно отнести следующие:

\begin{enumerate} 
	\item
	Магнитная система машины иенасыщена, в силу чего индуктивности машины не зависят от н.~с. (или токов); величины самих индуктивностей при этом определяются для некоторого значения магнитной проницаемости стали магнитопровода.
	\item
	Вместо действительных кривых распределения н.~с. и индукции в воздушном зазоре по расточке статора принимают только их основные, первые гармонические, соответственно чему наведенные в статоре э.~д.~с. выражаются синусоидами основной частоты.
	\item
	В магнитной системе машины отсутствуют какие-либо потери.
	\item
	Конструктивное выполнение машины обеспечивает полную симметрию фазных обмоток статора. Равным образом ротор также симметричен относительно своих продольной и поперечной осей.
	\item
	Предполагается, что как специально созданная продольная демпферная обмотка, так и все прочие естественные демпферные контуры, которые могут быть в продольной оси ротора, заменены одной эквивалентной продольной демпферной обмоткой; аналогично предполагается, что в поперечной оси ротора также имеется только одна эквивалентная поперечная демпферная обмотка\footnote{Для турбогенераторов при более точном анализе требуется учет нескольких демпферных контуров в каждой оси ротора.}.
	\item
	Скорость вращения ротора машины в течение рассматриваемого переходного процесса постоянна и равна синхронной.
\end{enumerate}

Даже для такой идеализированной машины анализ переходного процесса сопряжен со значительными трудностями, для преодоления которых приходится идти еще на некоторые упрощения. Сущность последних будет указана по ходу изложения.

Математические выкладки при учете демпферных обмоток значительно сложнее, и за громоздкостью получающихся выражений труднее понять их физический смысл. Поэтому вначале ограничимся рассмотрением машины без демпферных обмоток. Учет последних сделаем позднее, при этом для упрощения отступим от строгости самих выкладок и используем уже полученные в \href{chap:4}{гл.~4} результаты.

\section{Исходные уравнения}
\label{sec:7-2}


\section{Индуктивности обмоток синхронной машины}
\label{sec:7-3}


\section{Обобщенный вектор трехфазной системы}
\label{sec:7-4}


\section{Замена переменных}
\label{sec:7-5}


\section{Преобразование уравнений}
\label{sec:7-6}


\section{Выражения в операторной форме}
\label{sec:7-7}